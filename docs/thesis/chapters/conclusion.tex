\chapter{Zaključak}
\label{ch:conclusion}

\selectlanguage{croatian}

Ovaj rad uspješno je implementirao funkcionalan RAG sustav za analizu metapodataka otvorenih skupova podataka. Sustav je testiran na EU Open Data Portal endpointu i pokazuje obećavajuće rezultate za praktičnu primjenu. Glavni rezultat je da sustav može generirati ispravne SPARQL upite iz prirodnog jezika s uspješnošću od oko 90\% za dobro strukturirane upite, što je značajno poboljšanje u odnosu na tradicionalne pristupe.

Sustav je implementiran kao modularna arhitektura s jasno definiranim sučeljima između komponenti. ChromaDB vektorska baza omogućuje brzo pretraživanje sličnih primjera upita (prosječno vrijeme odziva 0.8 sekundi), dok Sentence Transformers model generira kvalitetne embeddinge za semantičko pretraživanje. LangChain agent orkestrira različite strategije pretraživanja i omogućuje multimodalni pristup (SPARQL, REST API, similarity search). OpenAI GPT-4 model pokazuje dobre rezultate u generiranju SPARQL upita kada ima dovoljno kontekstualnih informacija.

Automatska ekstrakcija DCAT/VoID sheme funkcionira pouzdano i omogućuje dinamičko prilagođavanje sustava promjenama u strukturi podataka. Sustav može identificirati preko 50 klasa i 100 svojstava s njihovim statistikama korištenja, što omogućuje generiranje upita svjesno sheme. Validacija upita implementirana je kroz dvostupanjski proces (sintaksna i semantička provjera) koji sprječava izvršavanje neispravnih upita.

Glavni problemi koji su identificirani tijekom razvoja uključuju ovisnost o komercijalnim LLM API-jima (latencija, troškovi), ograničenja tokena za složene upite, te potrebu za kontinuiranim održavanjem vektorske baze primjera. Sustav također pokazuje varijabilne performanse ovisno o složenosti upita -- jednostavni upiti poput ``klimatski podaci'' rade gotovo savršeno, dok složeni upiti kroz više domena mogu zahtijevati dodatno rukovanje greškama.

Za budući razvoj preporučuje se implementacija lokalnih LLM-ova~\cite{wang2023efficient}, proširivanje podrške za više jezika, optimizacija algoritama vektorskog pretraživanja~\cite{wang2023vector} za veće kolekcije podataka, te razvoj naprednijih strategija sinteze rezultata iz više izvora i multimodalnih modela~\cite{liu2023survey}. Automatsko generiranje SPARQL upita može se dodatno unaprijediti korištenjem najnovijih pristupa~\cite{emonet2024llm}. Sustav je dizajniran da podržava lako proširivanje i prilagodbu drugim portalima otvorenih podataka.

Kôd je dostupan u \texttt{src/} direktoriju s jasnom strukturom: \texttt{rag\_system.py} (glavna RAG logika), \texttt{schema\_extractor.py} (ekstrakcija sheme), \texttt{unified\_data\_assistant.py} (orkestracija), te \texttt{test/} direktorij s primjerima korištenja. Dokumentacija uključuje setup upute, API specifikacije i primjere integracije. Sustav je spreman za produkcijsku primjenu s dodatnim optimizacijama i proširenjima prema potrebi. 