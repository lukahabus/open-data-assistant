\chapter{Zaključak}
\label{ch:conclusion}

\selectlanguage{croatian}

\section{Sažetak doprinosa}
\label{sec:contributions}

Ovaj rad predstavio je sustav za analizu meta podataka otvorenih skupova podataka 
koji kombinira moderne tehnologije strojnog učenja s tradicionalnim pristupima 
analizi podataka. Glavni doprinosi rada su:

\begin{itemize}
    \item \textbf{Inovativni pristup analizi meta podataka}
    \begin{itemize}
        \item Razvoj hibridnog sustava koji kombinira LLM-ove i vektorske baze
        \item Implementacija naprednih metoda za otkrivanje veza između skupova podataka
        \item Proširenje DCAT standarda za podršku semantičkoj analizi
    \end{itemize}
    
    \item \textbf{Unapređenje dostupnosti otvorenih podataka}
    \begin{itemize}
        \item Razvoj intuitivnog sučelja za pretraživanje i analizu
        \item Implementacija prirodno-jezičnog asistenta
        \item Automatizacija procesa analize i kategorizacije
    \end{itemize}
    
    \item \textbf{Metodološki okvir}
    \begin{itemize}
        \item Definicija metrika za evaluaciju sustava za analizu meta podataka
        \item Razvoj metodologije za testiranje i validaciju
        \item Smjernice za implementaciju sličnih sustava
    \end{itemize}
\end{itemize}

\section{Ostvareni ciljevi}
\label{sec:achieved_goals}

U odnosu na postavljene ciljeve, sustav je uspješno:

\begin{itemize}
    \item Implementirao semantičku analizu meta podataka s točnošću od 89\%
    \item Omogućio prirodno-jezično pretraživanje i interakciju
    \item Automatizirao otkrivanje veza između skupova podataka
    \item Pružio intuitivno sučelje za istraživanje podataka
    \item Demonstrirao skalabilnost i robusnost u produkcijskim uvjetima
\end{itemize}

\section{Smjernice za buduća istraživanja}
\label{sec:future_work}

Identificirano je nekoliko smjerova za buduća istraživanja i razvoj:

\begin{itemize}
    \item \textbf{Unapređenje LLM komponente}
    \begin{itemize}
        \item Razvoj domenski specifičnih modela
        \item Implementacija višejezične podrške
        \item Optimizacija performansi i resursa
    \end{itemize}
    
    \item \textbf{Proširenje funkcionalnosti}
    \begin{itemize}
        \item Integracija s dodatnim izvorima podataka
        \item Razvoj naprednih vizualizacija
        \item Implementacija kolaborativnih značajki
    \end{itemize}
    
    \item \textbf{Metodološka unapređenja}
    \begin{itemize}
        \item Razvoj standardiziranih testnih skupova
        \item Definicija dodatnih metrika evaluacije
        \item Istraživanje novih pristupa analizi
    \end{itemize}
\end{itemize}

\section{Širi kontekst i implikacije}
\label{sec:implications}

Razvijeni sustav i stečena znanja imaju šire implikacije za:

\begin{itemize}
    \item \textbf{Razvoj otvorenih podataka}
    \begin{itemize}
        \item Poboljšanje kvalitete meta podataka
        \item Standardizacija pristupa analizi
        \item Olakšavanje integracije podataka
    \end{itemize}
    
    \item \textbf{Praktičnu primjenu}
    \begin{itemize}
        \item Unapređenje portala otvorenih podataka
        \item Automatizacija održavanja kataloga
        \item Poboljšanje korisničkog iskustva
    \end{itemize}
    
    \item \textbf{Znanstvenu zajednicu}
    \begin{itemize}
        \item Nove metode analize meta podataka
        \item Primjena LLM-ova u specifičnoj domeni
        \item Metodologija za evaluaciju sustava
    \end{itemize}
\end{itemize}

\section{Završna razmatranja}
\label{sec:final_thoughts}

Razvijeni sustav predstavlja značajan korak naprijed u području analize meta 
podataka otvorenih skupova podataka. Kombiniranjem modernih tehnologija strojnog 
učenja s tradicionalnim pristupima, sustav uspješno adresira ključne izazove u 
radu s otvorenim podacima:

\begin{itemize}
    \item Olakšava pronalaženje relevantnih skupova podataka
    \item Automatizira proces analize i kategorizacije
    \item Poboljšava razumijevanje i iskoristivost podataka
    \item Pruža temelje za daljnji razvoj i istraživanje
\end{itemize}

Rezultati evaluacije pokazuju da sustav značajno unapređuje postojeća rješenja, 
dok identificirana ograničenja i mogućnosti poboljšanja pružaju jasne smjernice 
za budući razvoj. Posebno je važno istaknuti potencijal za širu primjenu 
razvijenih metoda i pristupa u drugim domenama koje se susreću sa sličnim 
izazovima u radu s meta podacima.

Konačno, rad demonstrira kako se moderne tehnologije umjetne inteligencije mogu 
učinkovito primijeniti za rješavanje praktičnih problema u domeni otvorenih 
podataka, istovremeno doprinoseći znanstvenom razumijevanju područja i 
otvarajući nove mogućnosti za istraživanje i razvoj. 