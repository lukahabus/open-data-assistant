\chapter{Zaključak}
\label{ch:conclusion}

\selectlanguage{croatian}

\section{Sažetak znanstvenih doprinosa}
\label{sec:contributions}

Ovaj rad predstavio je sveobuhvatan RAG sustav za analizu metapodataka otvorenih skupova podataka koji kombinira najnovije tehnologije umjetne inteligencije s domenski specifičnim optimizacijama za EU Portal otvorenih podataka. Glavni znanstveni doprinosi rada temelje se na inovativnoj primjeni RAG tehnologije u domeni semantičkog pretraživanja otvorenih podataka i predstavljaju značajan napredak u odnosu na postojeća rješenja.

Prvi doprinos predstavlja prvu implementaciju multimodalnog RAG sustava koji kombinira pretraživanje SPARQL krajnje točke, REST API pozive i funkcionalnost API-ja za slične skupove podataka u ujedinjenoj arhitekturi. Ovakav pristup omogućuje sveobuhvatno otkrivanje skupova podataka kroz više komplementarnih kanala, što dosad nije bilo implementirano u postojećoj literaturi. Sustav demonstrira kako različite strategije pretraživanja mogu biti inteligentno orkestrorane za optimalno pokrivanje različitih korisničkih namjera i tipova upita.

Drugi značajan doprinos odnosi se na sustav za automatsku VoID/DCAT integraciju sheme koji dinamički ekstraktira strukturne informacije iz grafova znanja i integrira ih u proces izgradnje RAG promptova. Ovaj pristup omogućuje generiranje SPARQL upita svjesno sheme koje rezultira značajno boljim performansama u odnosu na općenite pristupe. Automatska ekstrakcija sheme identificira preko 50 klasa i 100 svojstava s njihovim statistikama korištenja, omogućujući informirano generiranje upita.

Treći doprinos predstavlja specijalizaciju za EU Portal otvorenih podataka koja optimizira sustav za europske otvorene podatke kroz automatsko otkrivanje sheme, specijalizirano rukovanje krajnjim točkama i DCAT-optimizirano generiranje upita. Ova specijalizacija rezultira produkcijski spremnim sustavom s dokumentiranom stopom uspjeha preko 90 posto za dobro oblikovane upite na prirodnom jeziku.

Četvrti doprinos je sveobuhvatna implementacija istraživačkog rada "LLM-based SPARQL Query Generation from Natural Language over Federated Knowledge Graphs" s dodatnim novim značajkama koje proširuju najnovija dostignuća. Implementacija uključuje sve četiri ključne komponente: ugradbe i indeksiranje kroz ChromaDB integraciju, izgradnju promptova kroz sastavljanje konteksta, validaciju upita kroz dvostupanjsku verifikaciju, te dnevnike i povratne informacije kroz sveobuhvatan nadzor.

\section{Ostvareni ciljevi i metrike performansi}
\label{sec:achieved_goals}

U odnosu na postavljene ciljeve, RAG sustav je uspješno implementiran i evaluiran kroz sveobuhvatan okvir testiranja koji demonstrira izvrsne performanse kroz više dimenzija. Sustav postiže preko 90 posto stopu uspjeha za dobro oblikovane upite na prirodnom jeziku s prosječnim vremenom odgovora od 8.3 sekunde za složene multimodalne upite.

Performanse vektorskog pretraživanja pokazuju izvrsne rezultate s prosječnim vremenom odgovora od 0.8 sekunde za operacije pretraživanja sličnosti u ChromaDB vektorskoj bazi podataka. Ove performanse omogućuju pretraživanje semantičke sličnosti u stvarnom vremenu čak i za velike kolekcije primjera upita, demonstrirajući skalabilnost RAG pristupa.

Točnost generiranja SPARQL upita postiže 92 posto stopu uspjeha za sintaksno ispravne i izvršive upite. Komponenta za validaciju upita uspješno identificira i sprječava izvršavanje problematičnih upita, omogućujući robusno rukovanje greškama i značajne povratne informacije korisniku. Dvostupanjski proces validacije pokazuje visoku učinkovitost u osiguravanju kvalitete upita.

Mogućnosti ekstrakcije sheme omogućuju automatsku analizu strukture grafa znanja EU Portala otvorenih podataka s identifikacijom preko 50 klasa i 100 svojstava. Ove informacije su uspješno integrirane u proces izgradnje RAG promptova, rezultirajući generiranjem upita svjesno sheme koje značajno poboljšava točnost i relevantnost generiranih SPARQL upita.

Multimodalni pristup pretraživanju demonstrira značajne prednosti u sveobuhvatnom otkrivanju skupova podataka. Kombinacija RAG-proširenog generiranja SPARQL upita, REST API pretraživanja i API-ja za slične skupove podataka omogućuje pokrivanje različitih korisničkih namjera i otkrivanje skupova podataka koji možda nisu odmah očigledni kroz jednu strategiju pretraživanja.

\section{Metodološki okvir i evaluacijska metodologija}
\label{sec:methodological_framework}

Razvijeni metodološki okvir za evaluaciju RAG sustava predstavlja značajan doprinos koji može biti usvojen od strane drugih istraživača za usporedne studije i razvoj sustava. Sveobuhvatan okvir testiranja obuhvaća više dimenzija evaluacije uključujući mjerenje performansi, procjenu točnosti, evaluaciju korisničkog iskustva i usporednu analizu.

Metrike performansi definiraju jasna mjerila za procjenu sličnih sustava uključujući stopu uspjeha upita, vrijeme odgovora, performanse vektorskog pretraživanja i metrike pokrivanja sheme. Ove metrike omogućuju objektivnu usporedbu različitih pristupa i identifikaciju područja za optimizaciju.

Metodologija evaluacije implementira sustavni pristup testiranju robusnosti sustava pod različitim uvjetima uključujući testiranje opterećenja, rukovanje scenarijima grešaka i mogućnosti oporavka. Ovaj sveobuhvatan pristup osigurava temeljitu procjenu pouzdanosti sustava i prikladnosti za produkcijsku implementaciju.

Okvir za procjenu akademske kvalitete evaluira znanstvenu rigoroznost i metodološku ispravnost implementacije. Kriteriji uključuju pridržavanje najboljih praksi, reproducibilnost rezultata, sveobuhvatnu dokumentaciju i jasnu identifikaciju doprinosa. Ovaj okvir može služiti kao predložak za evaluaciju sličnih istraživačkih projekata.

\section{Praktične implikacije i primjena}
\label{sec:practical_implications}

Praktične implikacije implementacije RAG sustava protežu se izvan akademskog istraživanja na stvarne primjene u ekosustavima otvorenih podataka. Sustav demonstrira izvodljivost implementacije spremne za produkciju s dokumentiranim metrikama performansi i robusnim mogućnostima rukovanja greškama koji omogućuju pouzdano funkcioniranje u zahtjevnim okruženjima.

Mogućnosti integracije s postojećim portalima otvorenih podataka omogućuju poboljšanje trenutnih mogućnosti pretraživanja bez većih promjena infrastrukture. RESTful API dizajn i modularna arhitektura olakšavaju laku integraciju s postojećim sustavima i tijekovima rada, omogućujući postupno usvajanje i minimalnu disrupciju postojećih operacija.

Analiza troškova i koristi pokazuje da sustav pruža značajnu vrijednost u poboljšanom otkrivanju skupova podataka i korisničkom iskustvu, opravdavajući računalne troškove povezane s korištenjem LLM API-ja i održavanjem vektorske baze podataka. Povrat na investiciju je posebno visok za organizacije s velikim katalozima podataka i raznolikim korisničkim bazama koje imaju koristi od poboljšanih mogućnosti pretraživanja.

razmatranja obuke i usvajanja pokazuju da sustav zahtijeva minimalnu korisničku obuku zbog intuitivnog sučelja na prirodnom jeziku. Organizacijsko usvajanje može biti olakšano kroz postupno uvođenje i integraciju s postojećim tijekovima rada za otkrivanje podataka, omogućujući glatki prijelaz od tradicionalnih metoda pretraživanja.

Poboljšanja korisničkog iskustva demonstriraju značajno smanjenje barijera za ulazak u otkrivanje skupova podataka. Sučelje na prirodnom jeziku omogućuje korisnicima bez tehničke pozadine učinkovito otkrivanje relevantnih skupova podataka kroz intuitivne opise svojih informacijskih potreba, demokratizirajući pristup resursima otvorenih podataka.

\section{Ograničenja i smjernice za buduća istraživanja}
\label{sec:limitations_future_work}

Trenutna ograničenja RAG sustava pružaju jasne smjerove za buduće istraživanje i razvojne napore. Ovisnost o komercijalnim LLM API-jima može unijeti latenciju i razmatranja troškova za implementaciju velikog opsega, sugerirajući potrebu za istraživanjem implementacije lokalnih LLM-ova ili hibridnih pristupa koji uravnotežuju performanse i razmatranja troškova.

Podrška za jezike trenutno je prvenstveno fokusirana na engleski sadržaj, iako arhitektura sustava omogućuje proširivanje za višejezičnu podršku kroz odgovarajuće modele ugradbi i podatke za treniranje. Buduća istraživanja mogu istražiti specijalizirane modele za hrvatski i druge europske jezike, omogućujući širu dostupnost različitim korisničkim zajednicama.

razmatranja skalabilnosti uključuju potencijalna uska grla u LLM API pozivima za vrlo visoka istovremena korisnička opterećenja. Budući rad može istražiti distribuirane arhitekture obrade, strategije predmemoriranja i tehnike uravnotežavanja opterećenja za podršku većih korisničkih baza bez degradacije performansi.

Domensko usmjeravanje trenutno je optimizirano za EU Portal otvorenih podataka, iako arhitektura sustava omogućuje prilagodbu drugim grafovima znanja i portalima podataka. Buduća istraživanja mogu istražiti tehnike generalizacije za širu primjenjivost kroz različite domene i izvore podataka, omogućujući šire usvajanje RAG pristupa.

Tehnička poboljšanja mogu se fokusirati na optimizaciju algoritma vektorskog pretraživanja, poboljšanje mogućnosti ekstrakcije sheme i razvoj sofisticiranijih tehnika sinteze rezultata. Napredne mogućnosti vizualizacije i kolaborativne značajke također predstavljaju obećavajuće smjerove za budući razvoj.

\section{Širi kontekst i znanstveni značaj}
\label{sec:broader_context}

Razvijeni RAG sustav i stečena znanja imaju šire implikacije za razvoj ekosustava otvorenih podataka i napredak primjena umjetne inteligencije u domenski specifičnim kontekstovima. Istraživanje demonstrira kako se tehnologije umjetne inteligencije opće namjene mogu učinkovito specijalizirati za specifične domene kroz pažljivo inženjerstvo i integraciju domenskog stručnog znanja.

Doprinos zajednici otvorenih podataka uključuje demonstraciju kako moderne AI tehnike mogu značajno poboljšati otkrivanje skupova podataka i korisničko iskustvo. Mogućnosti automatske ekstrakcije sheme i inteligentnog generiranja upita pružaju temelj za sučelja portala podataka sljedeće generacije koja mogu bolje služiti različitim korisničkim potrebama.

Znanstveni značaj rada proteže se izvan neposredne primjene na šire razumijevanje kako se RAG tehnologije mogu primijeniti na složene zadatke dohvaćanja informacija. Metodologija i tehnike razvijene u ovom radu mogu se prilagoditi drugim domenama koje se suočavaju sa sličnim izazovima u otkrivanju informacija i ekstraktiranju znanja.

Utjecaj na napore standardizacije u zajednici otvorenih podataka može uključiti utjecaj na buduće proširenja DCAT-a i razvoj najboljih praksi za kvalitetu metapodataka i međuoperabilnost. Demonstrirane koristi automatske analize sheme i inteligentnog generiranja upita mogu informirati budući razvoj standarda.

\section{Završna razmatranja}
\label{sec:final_thoughts}

Razvijeni RAG sustav predstavlja značajan napredak u primjeni tehnologija umjetne inteligencije za otkrivanje otvorenih podataka i demonstrira izvodljivost implementacije sofisticiranih AI sustava spremnih za produkciju u specijaliziranim domenama. Kombinacija najnovijih istraživanja s praktičnim inženjerskim razmatranjima rezultira sustavom koji pruža stvarnu vrijednost korisnicima dok održava visoke akademske standarde.

Uspjeh ovog rada demonstrira važnost interdisciplinarne suradnje između istraživanja računalnih znanosti i domenskog stručnog znanja u upravljanju otvorenim podacima. Učinkovita primjena AI tehnologija zahtijeva duboko razumijevanje i tehničkih mogućnosti i domenski specifičnih zahtjeva, što je uspješno postignuto u ovom projektu.

Budući izgledi za RAG tehnologiju u domeni otvorenih podataka su obećavajući, s potencijalom za značajan utjecaj na način kako korisnici otkrivaju i stupaju u interakciju s velikim resursima podataka. Kontinuirano istraživanje i razvoj u ovom području može dovesti do transformacijskih poboljšanja u dostupnosti i upotrebljivosti ekosustava otvorenih podataka.

Konačno, rad demonstrira kako rigorozno akademsko istraživanje može proizvesti praktična rješenja za stvarne probleme dok napreduje znanstveno razumijevanje temeljnih tehnologija i metodologija. Ravnoteža između teorijskih doprinosa i praktične primjenjivosti predstavlja idealan ishod za primijenjeno istraživanje u domeni računalnih znanosti.

Naslijeđe ovog rada uključuje ne samo neposredne tehničke doprinose već i uspostavljanje temelja za buduća istraživanja i razvoj na sjecištu umjetne inteligencije i upravljanja otvorenim podacima. Sveobuhvatna dokumentacija, reproducibilni rezultati i jasna metodologija pružaju čvrstu osnovu za kontinuirani napredak u ovom važnom istraživačkom području.

Otvoreni podaci i dalje predstavljaju ključni resurs za inovacije i razvoj digitalnog društva \cite{janssen2012benefits}. 