\chapter{Zaključak}
\label{ch:conclusion}

\selectlanguage{croatian}

Ovaj diplomski rad predstavio je razvoj i evaluaciju inteligentnog sustava za analizu metapodataka otvorenih skupova podataka. Kroz kombinaciju tehnika umjetne inteligencije, posebno RAG (\textit{Retrieval-Augmented Generation}) arhitekture, s tradicionalnim metodama obrade podataka, sustav uspješno premošćuje jaz između tehničke složenosti SPARQL jezika i intuitivnosti prirodnog jezika.

Implementirana RAG arhitektura demonstrira kako se veliki jezični modeli mogu kombinirati s vektorskim bazama podataka za stvaranje kontekstualno svjesnih i semantički točnih formalnih upita. Postignuta stopa uspjeha od 92\% za generiranje ispravnih SPARQL upita iz prirodnog jezika nadmašuje tradicionalne pristupe temeljene na ključnim riječima (65\%) i generičke LLM pristupe bez RAG-a (78\%). ChromaDB vektorska baza podataka pokazala se kao prikladna za semantičko pretraživanje, omogućavajući brzo dohvaćanje relevantnih primjera s prosječnim vremenom od 0.8 sekundi. Implementacija predmemoriranja dodatno poboljšava performanse, postižući 34\% pogodetka cache-a za česte upite.

Multimodalni pristup pretraživanju, koji kombinira RAG-generirane SPARQL upite, REST API pozive i pronalaženje sličnih skupova podataka, demonstrira prednosti hibridnog pristupa. Ova hibridna strategija osigurava stabilnost sustava i pokriva različite scenarije korištenja, od jednostavnih pretraživanja do složenih analitičkih upita. Sustav demokratizira pristup otvorenim podacima omogućavajući korisnicima bez specijaliziranog tehničkog znanja da postavljaju složena analitička pitanja.

Istraživačka pitanja postavljena na početku rada uspješno su riješena. RAG arhitektura može generirati sintaksno ispravne i semantički relevantne SPARQL upite iz prirodnog jezika s 92\% točnosti, gdje ključ uspjeha leži u kombinaciji semantičkog pretraživanja za dohvat relevantnih primjera i velikih jezičnih modela za generiranje upita prilagođenih specifičnom kontekstu. Vektorska baza podataka s reprezentativnim primjerima upita i relevantnim metapodacima o shemi uspješno je uspostavljena i pokazuje dobre performanse, omogućavajući brzo i precizno dohvaćanje relevantnih informacija. Multimodalni pristup koji kombinira SPARQL upite, REST API pozive i pronalaženje sličnih skupova podataka dokazano pruža bolju pokrivenost rezultata od bilo koje pojedinačne metode.

Sustav pokazuje prosječno vrijeme odziva od 8.3 sekunde po upitu, s mogućnošću opsluživanja do 10 istovremenih korisnika bez degradacije performansi. Implementacija predmemoriranja i optimizacija vektorskog pretraživanja omogućavaju prihvatljive performanse za male i srednje implementacije. Sustavna evaluacija performansi, točnosti i stabilnosti provedena je kroz sveobuhvatan skup od 100 testnih scenarija, pružajući jasnu sliku mogućnosti i ograničenja sustava.

Unatoč postignućima, važno je priznati ograničenja trenutne implementacije. Ovisnost o komercijalnim LLM API-jima uvodi pitanja troškova, latencije i privatnosti koja mogu ograničiti primjenu u određenim scenarijima. Mjesečni troškovi mogu varirati od 80 do 80,000 USD ovisno o broju korisnika i intenzitetu korištenja. Skalabilnost sustava, iako zadovoljavajuća za male i srednje implementacije, zahtijeva poboljšanja za podršku velikog broja istovremenih korisnika. Domensko usmjeravanje na EU Portal i DCAT standard, iako omogućava visoku preciznost, ograničava neposrednu primjenjivost na druge portale otvorenih podataka.

Razvijeni sustav ima značajan potencijal utjecaja na različite skupine korisnika. Istraživači i akademici mogu brže pronaći relevantne skupove podataka za svoje analize, novinari i analitičari mogu postavljati složena pitanja bez tehničkih predznanja, kreatori javnih politika mogu lakše pristupiti podacima potrebnim za informirane odluke, a građani mogu transparentnije pratiti javne podatke koji ih zanimaju.

Budući razvoj trebao bi se fokusirati na prelazak na lokalne LLM modele (Llama 3, CodeLlama) za smanjenje troškova i poboljšanje privatnosti, implementaciju distribuirane arhitekture za poboljšanje skalabilnosti, proširenje podrške na druge jezike, posebno nacionalne jezike EU, razvoj domenski specifičnih modela za različite tipove portala te integraciju naprednih vizualizacijskih alata. Prelazak na distribuiranu mikroservisnu arhitekturu omogućit će skaliranje na razinu nacionalnih i pan-europskih portala, dok će integracija vizualizacijskih i analitičkih alata pretvoriti sustav iz alata za pristup podacima u platformu za otkrivanje znanja.

Ovaj rad demonstrira da je moguće premostiti jaz između tehničke složenosti modernih portala otvorenih podataka i potreba običnih korisnika. Kombinacija umjetne inteligencije s tradicionalnim pristupima otvara nove mogućnosti za demokratizaciju pristupa javnim podacima. Razvijeni sustav predstavlja korak prema budućnosti u kojoj će pristup i analiza otvorenih podataka biti dostupni svima, neovisno o tehničkoj ekspertizi. Time se ostvaruje puni potencijal otvorenih podataka kao resursa za transparentnost, inovacije i društveni napredak. 