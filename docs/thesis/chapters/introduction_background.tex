\chapter{Uvod}
\label{ch:introduction_background}

\selectlanguage{croatian}

U suvremenom digitalnom okruženju, otvoreni podaci predstavljaju ključni resurs za transparentnost, istraživanja i inovacije. Unatoč postojanju milijuna skupova podataka na portalima poput EU Portala otvorenih podataka, njihova iskoristivost ostaje ograničena zbog tehničkih barijera. Korisnici se susreću s potrebom poznavanja SPARQL jezika za pretraživanje, ograničenjima tradicionalnog pretraživanja ključnih riječi te izazovima povezivanja više skupova podataka za složene analize.

Ovaj diplomski rad predstavlja razvoj i implementaciju inteligentnog sustava koji koristi napredne tehnike umjetne inteligencije za premošćivanje jaza između tehničke složenosti i korisničkih potreba. Sustav kombinira arhitekturu Retrieval-Augmented Generation (RAG) s velikim jezičnim modelima kako bi omogućio korisnicima postavljanje pitanja na prirodnom jeziku i automatsko generiranje odgovarajućih SPARQL upita.

Ključna inovacija rada leži u implementaciji vektorske baze podataka koja semantički indeksira primjere upita i metapodatke o strukturi podataka, omogućavajući sustavu da učini iz prethodnih iskustava i konteksta. Sustav implementira multimodalni pristup koji kombinira tri različite metode dohvaćanja podataka: RAG-generirane SPARQL upite, REST API pozive i pretragu sličnih skupova podataka, čime se osigurava robusnost i sveobuhvatnost rezultata.

Evaluacija sustava provedena na EU Portalu otvorenih podataka pokazuje značajne rezultate. Sustav postiže 92\% točnost u generiranju ispravnih SPARQL upita iz prirodnog jezika, što predstavlja značajno poboljšanje u odnosu na tradicionalne pristupe temeljene na ključnim riječima (65\%) i generičke pristupe velikih jezičnih modela bez RAG arhitekture (78\%). Prosječno vrijeme odziva od 8.3 sekunde omogućava praktičnu upotrebu, dok vektorsko pretraživanje relevantnih primjera traje manje od sekunde.

Testiranje na skupu od 100 različitih scenarija upita potvrdilo je stabilnost sustava na različite tipove zahtjeva, od jednostavnih pretraživanja do složenih analitičkih pitanja koja zahtijevaju povezivanje više skupova podataka. Sustav uspješno rukuje mrežnim prekidima, API ograničenjima i različitim tipovima grešaka, čineći ga pogodnim za praktičnu upotrebu.

Praktični doprinos rada ogleda se u demokratizaciji pristupa otvorenim podacima. Sustav omogućava istraživačima, novinarima, kreatorima javnih politika i građanima da bez tehničkih predznanja postavljaju složena pitanja i dobivaju relevantne odgovore. Time se proširuje krug korisnika otvorenih podataka i povećava njihova društvena vrijednost.

Tehnički doprinosi uključuju razvoj specijalizirane RAG arhitekture za SPARQL generiranje, implementaciju semantičkog indeksiranja metapodataka DCAT standarda, hibridni multimodalni pristup pretraživanju te sveobuhvatan sustav validacije i optimizacije performansi. Sustav je dizajniran kao modularan i proširiv, omogućavajući buduće prilagodbe različitim portalima i domenama.