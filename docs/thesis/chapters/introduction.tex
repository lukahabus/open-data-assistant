\chapter{Uvod}
\label{ch:introduction}

\selectlanguage{croatian}

\section{Motivacija}
\label{sec:motivation}

U današnje vrijeme, otvoreni podaci predstavljaju ključan resurs za inovacije, 
transparentnost i razvoj digitalnog društva. Javne institucije, istraživačke organizacije 
i privatni sektor objavljuju sve veće količine otvorenih podataka. Međutim, sama 
dostupnost podataka ne jamči njihovu iskoristivost. Jedan od ključnih izazova je 
učinkovito pronalaženje, razumijevanje i povezivanje različitih skupova podataka.

Normirani formati meta podataka, poput DCAT-a (Data Catalog Vocabulary), omogućuju 
standardizirani opis skupova podataka. No, tradicionalni pristupi pretraživanju i analizi 
meta podataka često su ograničeni na jednostavno tekstualno podudaranje ili preddefinirane 
kategorije. Ovo ograničava mogućnosti otkrivanja skrivenih veza između skupova podataka 
i otežava korisnicima pronalaženje relevantnih informacija.

\section{Ciljevi rada}
\label{sec:objectives}

Glavni cilj ovog rada je razvoj sustava koji će unaprijediti iskoristivost otvorenih 
podataka kroz naprednu analizu njihovih meta podataka. Specifični ciljevi uključuju:

\begin{itemize}
    \item Istraživanje mogućnosti velikih jezičnih modela u kontekstu analize meta podataka
    \item Razvoj sustava za semantičku analizu i povezivanje skupova podataka
    \item Implementaciju intuitivnog korisničkog sučelja za interakciju s meta podacima
    \item Integraciju s postojećim portalima otvorenih podataka (CKAN)
    \item Evaluaciju učinkovitosti i korisnosti predloženog rješenja
\end{itemize}

\section{Doprinosi}
\label{sec:contributions}

Glavni doprinosi ovog rada su:

\begin{enumerate}
    \item Nova metodologija za semantičku analizu meta podataka korištenjem velikih jezičnih modela
    \item Prošireni DCAT model s podrškom za semantičke veze između skupova podataka
    \item Implementacija sustava koji demonstrira praktičnu primjenu predložene metodologije
    \item Empirijska evaluacija učinkovitosti sustava u stvarnim uvjetima
\end{enumerate}

\section{Struktura rada}
\label{sec:structure}

Rad je organiziran na sljedeći način:

\begin{itemize}
    \item Poglavlje \ref{ch:background} predstavlja teorijsku podlogu rada, uključujući 
    pregled otvorenih podataka, DCAT standarda i velikih jezičnih modela.
    
    \item Poglavlje \ref{ch:related_work} daje pregled postojećih rješenja i 
    istraživanja u području analize meta podataka.
    
    \item Poglavlje \ref{ch:system_design} opisuje arhitekturu i dizajn predloženog 
    sustava.
    
    \item Poglavlje \ref{ch:implementation} detaljno predstavlja implementaciju sustava, 
    uključujući ključne komponente i tehnička rješenja.
    
    \item Poglavlje \ref{ch:evaluation} prikazuje rezultate evaluacije sustava i 
    diskusiju o njegovoj učinkovitosti.
    
    \item Poglavlje \ref{ch:conclusion} donosi zaključke rada i smjernice za buduća 
    istraživanja.
\end{itemize}