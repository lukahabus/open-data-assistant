\chapter{Uvod}
\label{ch:introduction}

\selectlanguage{croatian}

\section{Motivacija}
\label{sec:motivation}

U suvremenom digitalnom društvu, otvoreni podaci predstavljaju temeljni resurs za inovacije, transparentnost javnih institucija i demokratski razvoj \cite{janssen2012benefits, charalabidis2018open, bizer2009linked}. Europska unija kroz svoj Portal otvorenih podataka (EU Open Data Portal) omogućuje pristup milijunima skupova podataka \cite{wilson2013developing}. Međutim, sama dostupnost podataka ne jamči njihovu učinkovitu iskoristivost, što predstavlja značajan izazov za istraživače, analitičare i kreatore javnih politika.

Tradicionalni pristupi pretraživanja otvorenih podataka često su ograničeni na jednostavno tekstualno poklapanje ključnih riječi ili navigaciju kroz unaprijed definirane kategorije. Ovakvi pristupi ne omogućuju semantičko razumijevanje sadržaja podataka niti otkrivanje skrivenih veza između različitih skupova podataka. Korisnici se često suočavaju s poteškoćama pri formuliranju preciznih SPARQL upita potrebnih za pristup strukturiranim podacima, što ograničava široku adopciju i korištenje dostupnih resursa.

Nedavni napretci u području velikih jezičnih modela (LLM) i tehnika povratnog dohvaćanja i generiranja (Retrieval-Augmented Generation - RAG) otvaraju nove mogućnosti za intuitivno pretraživanje i analizu strukturiranih podataka \cite{brown2020language, lewis2020retrieval, liu2023survey}. Kombiniranjem semantičkih vektorskih reprezentacija s kontekstualnim generiranjem odgovora, moguće je stvoriti sustave koji omogućuju prirodno jezično upućivanje upita nad složenim grafovima znanja.

\section{Problem istraživanja}
\label{sec:research_problem}

Ovo istraživanje fokusira se na temeljni problem pristupačnosti i iskoristivosti otvorenih podataka dostupnih kroz SPARQL krajnje točke. Postojeći sustavi zahtijevaju od korisnika poznavanje formalnih jezika za upite kao što je SPARQL, što predstavlja značajnu barijeru za širu adopciju. Dodatno, korisnici često ne posjeduju dovoljno znanja o strukturi i sadržaju dostupnih podataka kako bi formulirali efikasne upite.

Specifični problemi koje istraživanje adresira uključuju nepostojanje intuitivnih sučelja za semantičko pretraživanje nad federiranim grafovima znanja, ograničenu mogućnost otkrivanja povezanih skupova podataka kroz različite domene, te nedostatak inteligentnih sustava koji mogu razumjeti korisničke namjere izražene prirodnim jezikom i transformirati ih u precizne formalne upite.

\section{Ciljevi rada}
\label{sec:objectives}

Glavni cilj ovog rada predstavlja razvoj i implementaciju naprednog sustava za analizu metapodataka otvorenih skupova podataka koristeći RAG tehnologiju. Sustav omogućuje prirodno jezično upućivanje upita nad EU Portalom otvorenih podataka kroz kombinaciju vektorskog pretraživanja sličnosti, kontekstualnog generiranja SPARQL upita i multimodalnog pristupa dohvaćanju podataka.

Prvi specifični cilj usmjeren je na implementaciju RAG arhitekture koja kombinira ChromaDB vektorsku bazu podataka s Sentence Transformers modelima za generiranje semantičkih ugradbi pitanja i odgovora. Sustav koristi all-MiniLM-L6-v2 model za stvaranje vektorskih reprezentacija dimenzije 384, omogućujući visokokvalitetno semantičko pretraživanje sličnosti.

Drugi cilj fokusira se na razvoj automatskog sustava za ekstrakciju i analizu informacija o shemi iz SPARQL krajnjih točaka. Sustav automatski dohvaća VoID (Vocabulary of Interlinked Datasets) deskriptore i analizira DCAT (Data Catalog Vocabulary) strukturu, identificirajući preko 50 klasa i 100 svojstava s njihovim statistikama korištenja.

Treći cilj obuhvaća implementaciju ujedinjenog asistenta za podatke koji orkestrira više komplementarnih pristupa pretraživanja podataka. Sustav kombinira RAG-prošireno generiranje SPARQL upita, REST API pozive i funkcionalnost API-ja za slične skupove podataka kroz inteligentnu arhitekturu temeljenu na agentima.

Četvrti cilj usmjeren je na validaciju i evaluaciju sustava kroz sveobuhvatan okvir testiranja koji demonstrira postizanje stope uspjeha veće od 90 posto za dobro oblikovane upite na prirodnom jeziku, s prosječnim vremenom odgovora od 8-15 sekundi za složene multimodalne upite.

\section{Znanstveni doprinos}
\label{sec:contributions}

Glavni znanstveni doprinosi ovog rada temelje se na inovativnoj primjeni RAG tehnologije u domeni semantičkog pretraživanja otvorenih podataka i predstavljaju značajan napredak u odnosu na postojeća rješenja.

Prvi doprinos predstavlja prvu implementaciju multimodalnog RAG sustava koji kombinira pretraživanje SPARQL krajnjih točaka, REST API pozive i funkcionalnost API-ja za slične skupove podataka u ujedinjenoj arhitekturi. Ovakav pristup omogućuje sveobuhvatno otkrivanje skupova podataka kroz više komplementarnih kanala, što dosad nije bilo implementirano u postojećoj literaturi.

Drugi značajan doprinos odnosi se na sustav automatske VoID/DCAT integracije sheme koji dinamički ekstraktira strukturne informacije iz grafova znanja i integrira ih u proces izgradnje RAG promptova. Ovaj pristup omogućuje generiranje SPARQL upita svjesno sheme koje rezultira značajno boljim performansama u odnosu na općenite pristupe.

Treći doprinos predstavlja specijalizaciju za EU Portal otvorenih podataka koja optimizira sustav za europske otvorene podatke kroz automatsko otkrivanje sheme, specijalizirano rukovanje krajnjim točkama i DCAT-optimizirano generiranje upita. Ova specijalizacija rezultira produkcijski spremnim sustavom s dokumentiranom stopom uspjeha preko 90 posto.

Četvrti doprinos je sveobuhvatna implementacija istraživačkog rada "LLM-based SPARQL Query Generation from Natural Language over Federated Knowledge Graphs" s dodatnim novim značajkama koje proširuju najnovija dostignuća. Implementacija uključuje sve četiri ključne komponente: ugradbe i indeksiranje, izgradnju promptova, validaciju upita te dnevnike i povratne informacije, s dokumentiranim metrikama performansi.

\section{Struktura rada}
\label{sec:structure}

Rad je organiziran kroz šest glavnih poglavlja koja sustavno prezentiraju teorijsku podlogu, dizajn sustava, implementacijske detalje, eksperimentalnu evaluaciju i zaključne napomene.

Poglavlje \ref{ch:background} predstavlja sveobuhvatan pregled teorijske podloge rada. Poglavlje pokriva temeljne koncepte otvorenih podataka i DCAT standarda, detaljnu analizu RAG tehnologije i velikih jezičnih modela, te pregled semantičkih web tehnologija i SPARQL jezika za upite. Dodatno, poglavlje uključuje detaljnu analizu referencirane literature i najnovijih dostignuća u domeni prijevoda prirodnog jezika u SPARQL.

Poglavlje \ref{ch:system_design} opisuje sveobuhvatnu arhitekturu predloženog sustava kroz perspektivu više slojeva. Poglavlje detaljno prezentira arhitekturu RAG sustava s ChromaDB integracijom, dizajnski obrazac ujedinjenog asistenta za podatke, multimodalni pristup upitima te metodologiju automatske ekstrakcije sheme. Uključeni su arhitekturni dijagrami i detaljne specifikacije komponenti.

Poglavlje \ref{ch:implementation} predstavlja detaljnu tehničku implementaciju svih ključnih komponenti sustava. Poglavlje pokriva implementacijske detalje RAG sustava, razvoj ekstraktora sheme, implementaciju ujedinjenog asistenta za podatke, sveobuhvatne mehanizme validacije i rukovanja greškama, te strategije optimizacije performansi. Sve implementacijske detalje podržane su stvarnim primjerima koda i tehničkim specifikacijama.

Poglavlje \ref{ch:evaluation} prikazuje sveobuhvatnu eksperimentalnu evaluaciju sustava kroz više dimenzija evaluacije. Poglavlje uključuje detaljno mjerenje performansi, usporednu analizu različitih pristupa, evaluaciju korisničkog iskustva te procjenu akademske kvalitete. Svi rezultati prezentirani su s odgovarajućom statističkom analizom i interpretacijom.

Poglavlje \ref{ch:conclusion} donosi sveobuhvatne zaključke rada, sustavni sažetak glavnih postignuća, diskusiju ograničenja trenutnog pristupa, te detaljnu mapu budućih istraživanja u domeni inteligentnih sustava za otkrivanje podataka.