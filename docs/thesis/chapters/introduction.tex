\chapter{Uvod}
\label{ch:introduction}

\selectlanguage{croatian}

\section{Motivacija}
\label{sec:motivation}

U današnje vrijeme, otvoreni podaci predstavljaju ključan resurs za inovacije, 
transparentnost i razvoj digitalnog društva. Javne institucije, istraživačke organizacije 
i privatni sektor objavljuju sve veće količine otvorenih podataka. Međutim, sama 
dostupnost podataka ne jamči njihovu iskoristivost. Jedan od ključnih izazova je 
učinkovito pronalaženje, razumijevanje i povezivanje različitih skupova podataka.

Normirani formati meta podataka, poput DCAT-a (Data Catalog Vocabulary), omogućuju 
standardizirani opis skupova podataka. No, tradicionalni pristupi pretraživanju i analizi 
meta podataka često su ograničeni na jednostavno tekstualno podudaranje ili preddefinirane 
kategorije. Ovo ograničava mogućnosti otkrivanja skrivenih veza između skupova podataka 
i otežava korisnicima pronalaženje relevantnih informacija.

\section{Ciljevi rada}
\label{sec:objectives}

Glavni cilj ovog rada je razvoj sustava koji će unaprijediti iskoristivost otvorenih 
podataka kroz naprednu analizu njihovih meta podataka. Specifični ciljevi uključuju:

\begin{itemize}
    \item Istraživanje mogućnosti velikih jezičnih modela u kontekstu analize meta podataka
    \item Razvoj sustava za semantičku analizu i povezivanje skupova podataka
    \item Implementaciju intuitivnog korisničkog sučelja za interakciju s meta podacima
    \item Integraciju s postojećim portalima otvorenih podataka (CKAN)
    \item Evaluaciju učinkovitosti i korisnosti predloženog rješenja
\end{itemize}

\section{Doprinosi}
\label{sec:contributions}

Glavni doprinosi ovog rada su:

\begin{enumerate}
    \item Nova metodologija za semantičku analizu meta podataka korištenjem velikih jezičnih modela
    \item Prošireni DCAT model s podrškom za semantičke veze između skupova podataka
    \item Implementacija sustava koji demonstrira praktičnu primjenu predložene metodologije
    \item Empirijska evaluacija učinkovitosti sustava u stvarnim uvjetima
\end{enumerate}

\section{Struktura rada}
\label{sec:structure}

Rad je organiziran na sljedeći način:

\begin{itemize}
    \item Poglavlje \ref{ch:background} predstavlja teorijsku podlogu rada, uključujući 
    pregled otvorenih podataka, DCAT standarda i velikih jezičnih modela.
    
    \item Poglavlje \ref{ch:related_work} daje pregled postojećih rješenja i 
    istraživanja u području analize meta podataka.
    
    \item Poglavlje \ref{ch:system_design} opisuje arhitekturu i dizajn predloženog 
    sustava.
    
    \item Poglavlje \ref{ch:implementation} detaljno predstavlja implementaciju sustava, 
    uključujući ključne komponente i tehnička rješenja.
    
    \item Poglavlje \ref{ch:evaluation} prikazuje rezultate evaluacije sustava i 
    diskusiju o njegovoj učinkovitosti.
    
    \item Poglavlje \ref{ch:conclusion} donosi zaključke rada i smjernice za buduća 
    istraživanja.
\end{itemize}

\selectlanguage{english}

% English version of the introduction
\chapter*{Introduction}
\addcontentsline{toc}{chapter}{Introduction (English)}

\section*{Motivation}
In today's world, open data represents a key resource for innovation, transparency, and 
the development of digital society. Public institutions, research organizations, and the 
private sector are publishing increasing amounts of open data. However, data availability 
alone does not guarantee its usability. One of the key challenges is effectively finding, 
understanding, and connecting different datasets.

Standardized metadata formats, such as DCAT (Data Catalog Vocabulary), enable standardized 
description of datasets. However, traditional approaches to searching and analyzing metadata 
are often limited to simple text matching or predefined categories. This limits the 
possibilities of discovering hidden connections between datasets and makes it difficult 
for users to find relevant information.

\section*{Objectives}
The main objective of this thesis is to develop a system that will enhance the usability 
of open data through advanced analysis of their metadata. Specific objectives include:

\begin{itemize}
    \item Exploring the capabilities of large language models in the context of metadata analysis
    \item Developing a system for semantic analysis and linking of datasets
    \item Implementing an intuitive user interface for metadata interaction
    \item Integration with existing open data portals (CKAN)
    \item Evaluating the effectiveness and usefulness of the proposed solution
\end{itemize}

\section*{Contributions}
The main contributions of this work are:

\begin{enumerate}
    \item A new methodology for semantic metadata analysis using large language models
    \item Extended DCAT model with support for semantic relationships between datasets
    \item Implementation of a system demonstrating practical application of the proposed methodology
    \item Empirical evaluation of system effectiveness in real conditions
\end{enumerate}

\section*{Structure}
The thesis is organized as follows:

\begin{itemize}
    \item Chapter \ref{ch:background} presents the theoretical background, including an 
    overview of open data, DCAT standard, and large language models.
    
    \item Chapter \ref{ch:related_work} provides an overview of existing solutions and 
    research in the field of metadata analysis.
    
    \item Chapter \ref{ch:system_design} describes the architecture and design of the 
    proposed system.
    
    \item Chapter \ref{ch:implementation} presents the system implementation in detail, 
    including key components and technical solutions.
    
    \item Chapter \ref{ch:evaluation} shows the results of system evaluation and 
    discusses its effectiveness.
    
    \item Chapter \ref{ch:conclusion} provides conclusions and directions for future 
    research.
\end{itemize} 