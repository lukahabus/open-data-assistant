\chapter{Implementacija}
\label{ch:implementation}

\selectlanguage{croatian}

\section{Tehnologije i alati}
\label{sec:technologies}

Za implementaciju sustava odabrane su moderne tehnologije koje omogućuju razvoj 
skalabilnog i održivog rješenja. Odabir tehnologija vođen je sljedećim kriterijima:
\begin{itemize}
    \item Zrelost i stabilnost tehnologije
    \item Dostupnost dokumentacije i zajednice
    \item Performanse i skalabilnost
    \item Jednostavnost integracije
\end{itemize}

\subsection{Backend tehnologije}
\begin{itemize}
    \item \textbf{Python} - glavni programski jezik
    \begin{itemize}
        \item FastAPI za REST API
        \item Pydantic za validaciju podataka
        \item SQLAlchemy za rad s bazom podataka
    \end{itemize}
    
    \item \textbf{Vektorska baza} - Faiss/Milvus
    \begin{itemize}
        \item Pohrana i pretraživanje embeddings
        \item Visoke performanse za similarity search
        \item Skalabilnost za velike količine podataka
    \end{itemize}
    
    \item \textbf{LLM integracija}
    \begin{itemize}
        \item OpenAI API
        \item LangChain za orkestraciju
        \item Sentence Transformers za embeddings
    \end{itemize}
\end{itemize}

\subsection{Frontend tehnologije}
\begin{itemize}
    \item \textbf{React} - biblioteka za korisničko sučelje
    \begin{itemize}
        \item TypeScript za type safety
        \item Material-UI za komponente
        \item React Query za state management
    \end{itemize}
    
    \item \textbf{Vizualizacija}
    \begin{itemize}
        \item D3.js za grafove
        \item Chart.js za statistike
        \item React Flow za interaktivne dijagrame
    \end{itemize}
\end{itemize}

\section{Implementacija komponenti}
\label{sec:components}

\subsection{DCAT Adapter}
Implementacija DCAT adaptera uključuje:
\begin{itemize}
    \item Parsiranje različitih formata meta podataka
    \item Normalizaciju u standardni format
    \item Validaciju prema DCAT shemi
    \item Proširenja za dodatne metrike
\end{itemize}

\begin{lstlisting}[language=Python, caption=Implementacija DCAT adaptera]
class DCATAdapter:
    def __init__(self):
        self.validator = DCATValidator()
        self.normalizer = MetadataNormalizer()
    
    def process_metadata(self, metadata: dict) -> DCATDataset:
        normalized = self.normalizer.normalize(metadata)
        validated = self.validator.validate(normalized)
        return DCATDataset.from_dict(validated)
\end{lstlisting}

\subsection{Embedding Engine}
Modul za generiranje i upravljanje vektorskim reprezentacijama:
\begin{itemize}
    \item Korištenje predtreniranih modela
    \item Prilagodba za specifične domene
    \item Optimizacija performansi
    \item Cachiranje rezultata
\end{itemize}

\begin{lstlisting}[language=Python, caption=Implementacija Embedding Engine-a]
class EmbeddingEngine:
    def __init__(self, model_name: str = "all-MiniLM-L6-v2"):
        self.model = SentenceTransformer(model_name)
        self.cache = EmbeddingCache()
    
    def generate_embeddings(self, text: str) -> np.ndarray:
        if cached := self.cache.get(text):
            return cached
        embedding = self.model.encode(text)
        self.cache.store(text, embedding)
        return embedding
\end{lstlisting}

\subsection{Semantic Analyzer}
Komponenta za semantičku analizu implementira:
\begin{itemize}
    \item Analizu sličnosti skupova podataka
    \item Otkrivanje tematskih klastera
    \item Generiranje semantičkih veza
    \item Procjenu kvalitete meta podataka
\end{itemize}

\begin{lstlisting}[language=Python, caption=Implementacija Semantic Analyzer-a]
class SemanticAnalyzer:
    def __init__(self, embedding_engine: EmbeddingEngine):
        self.embedding_engine = embedding_engine
        self.index = VectorIndex()
    
    def analyze_dataset(self, dataset: DCATDataset) -> Analysis:
        embeddings = self.generate_dataset_embeddings(dataset)
        similar = self.find_similar_datasets(embeddings)
        clusters = self.identify_clusters(embeddings)
        return Analysis(similar=similar, clusters=clusters)
\end{lstlisting}

\subsection{LLM Assistant}
Implementacija asistenta uključuje:
\begin{itemize}
    \item Integraciju s OpenAI API-jem
    \item Upravljanje kontekstom razgovora
    \item Generiranje prirodnih odgovora
    \item Analizu korisničkih upita
\end{itemize}

\begin{lstlisting}[language=Python, caption=Implementacija LLM Asistenta]
class LLMAssistant:
    def __init__(self, model: str = "gpt-3.5-turbo"):
        self.llm = ChatOpenAI(model=model)
        self.memory = ConversationBufferMemory()
    
    async def analyze_query(self, query: str) -> AssistantResponse:
        context = self.memory.get_context()
        response = await self.llm.generate_response(query, context)
        self.memory.add_interaction(query, response)
        return response
\end{lstlisting}

\section{API implementacija}
\label{sec:api}

REST API implementiran je korištenjem FastAPI okvira:

\begin{lstlisting}[language=Python, caption=Implementacija glavnih API endpointa]
@router.post("/query")
async def query_datasets(request: QueryRequest) -> AssistantResponse:
    response = await assistant.analyze_query(request.query)
    return response

@router.get("/datasets/{dataset_id}")
async def get_dataset(dataset_id: str) -> Dataset:
    dataset = await repository.get_dataset(dataset_id)
    return dataset

@router.post("/analyze")
async def analyze_dataset(request: AnalysisRequest) -> List[MetadataInsight]:
    insights = await analyzer.analyze_dataset(request.dataset_id)
    return insights
\end{lstlisting}

\section{Frontend implementacija}
\label{sec:frontend}

\subsection{Komponente korisničkog sučelja}
Implementirane su sljedeće React komponente:
\begin{itemize}
    \item \textbf{Dataset Explorer} - pregled i pretraživanje skupova podataka
    \item \textbf{Metadata Viewer} - detaljni prikaz meta podataka
    \item \textbf{Relationship Graph} - vizualizacija veza
    \item \textbf{Search Interface} - napredno pretraživanje
\end{itemize}

\begin{lstlisting}[language=TypeScript, caption=Implementacija Dataset komponente]
interface DatasetProps {
  dataset: Dataset;
  onAnalyze: (id: string) => void;
}

const Dataset: React.FC<DatasetProps> = ({ dataset, onAnalyze }) => {
  return (
    <Card>
      <CardHeader title={dataset.title} />
      <CardContent>
        <Typography>{dataset.description}</Typography>
        <MetadataList metadata={dataset.metadata} />
      </CardContent>
      <CardActions>
        <Button onClick={() => onAnalyze(dataset.id)}>
          Analyze
        </Button>
      </CardActions>
    </Card>
  );
};
\end{lstlisting}

\section{Testiranje}
\label{sec:testing}

\subsection{Jedinični testovi}
Implementirani su testovi za sve ključne komponente:

\begin{lstlisting}[language=Python, caption=Primjer unit testa]
def test_embedding_generation():
    engine = EmbeddingEngine()
    text = "Test dataset description"
    embedding = engine.generate_embeddings(text)
    
    assert embedding is not None
    assert embedding.shape == (384,)  # Expected dimension
    
def test_semantic_analysis():
    analyzer = SemanticAnalyzer(mock_embedding_engine)
    dataset = create_test_dataset()
    analysis = analyzer.analyze_dataset(dataset)
    
    assert len(analysis.similar) > 0
    assert len(analysis.clusters) > 0
\end{lstlisting}

\subsection{Integracijski testovi}
Testiranje integracije komponenti:

\begin{lstlisting}[language=Python, caption=Primjer integracijskog testa]
async def test_full_analysis_pipeline():
    # Setup test components
    adapter = DCATAdapter()
    engine = EmbeddingEngine()
    analyzer = SemanticAnalyzer(engine)
    
    # Test pipeline
    metadata = load_test_metadata()
    dataset = adapter.process_metadata(metadata)
    embeddings = engine.generate_embeddings(dataset.description)
    analysis = analyzer.analyze_dataset(dataset)
    
    # Verify results
    assert analysis.is_valid()
    assert len(analysis.insights) > 0
\end{lstlisting}

\section{Optimizacija}
\label{sec:optimization}

\subsection{Performanse}
Implementirane su sljedeće optimizacije:
\begin{itemize}
    \item Cachiranje embeddings
    \item Batch processing za LLM pozive
    \item Optimizacija vektorskih upita
    \item Lazy loading podataka
\end{itemize}

\subsection{Skalabilnost}
Sustav je pripremljen za skaliranje kroz:
\begin{itemize}
    \item Asinkrono procesiranje
    \item Distribuirani cache
    \item Load balancing
    \item Horizontalno skaliranje komponenti
\end{itemize} 