\chapter{Evaluacija i rezultati}
\label{ch:evaluation}

\selectlanguage{croatian}

\section{Metodologija evaluacije i testni podaci}

Testni podaci uključuju EU Portal otvorenih podataka kao primarni izvor s preko milijun skupova podataka. Korišten je sveobuhvatan skup testnih upita koji pokrivaju različite domene: okoliš, energija, zdravstvo, transport i ekonomski podaci. Upiti su dizajnirani da testiraju različite razine složenosti - od jednostavnih ključnih riječi do složenih analitičkih upita.

Unaprijed definirani primjeri upita korišteni su za validaciju RAG funkcionalnosti. Ovi primjeri su bili posebno važni jer su omogućili usporedbu s očekivanim rezultatima i identifikaciju područja gdje sustav možda ne funkcionira kako treba. Ukupno je testirano preko 100 različitih upita kroz različite scenarije.

Metrike evaluacije uključuju stopu uspjeha upita (postotak upita na prirodnom jeziku koji uspješno generiraju izvršive SPARQL upite), vrijeme odgovora (ukupno vrijeme za potpunu multimodalnu obradu), performanse vektorskog pretraživanja (vrijeme za operacije pretraživanja sličnosti) i metrike pokrivanja sheme (potpunost automatske ekstrakcije sheme).

\section{Rezultati performansi i točnosti}

Mjerenje performansi RAG sustava provedeno je kroz sustavno testiranje preko 100 testnih upita. Rezultati pokazuju da sustav postiže preko 90\% stopu uspjeha za dobro oblikovane upite na prirodnom jeziku s prosječnim vremenom odgovora od 8.3 sekunde za složene multimodalne upite.

Performanse vektorskog pretraživanja pokazuju izvrsne rezultate s prosječnim vremenom odziva od 0.8 sekunde za operacije pretraživanja sličnosti. ChromaDB trajno pohranjivanje omogućava dosljedne performanse kroz sesije s brzim vremenima pokretanja sustava i pouzdanim funkcioniranjem čak i za velike kolekcije primjera upita.

Točnost generiranja SPARQL upita postiže 92\% stopu uspjeha za sintaksno ispravne i izvršive upite. Komponenta za validaciju upita uspješno identificira i sprječava izvršavanje problematičnih upita, omogućavajući robusno rukovanje greškama i značajne povratne informacije korisniku. Dvostupanjski proces validacije pokazuje visoku učinkovitost u osiguravanju kvalitete upita.

Performanse ekstrakcije sheme pokazuju da sustav može automatski ekstraktirati preko 50 klasa i 100 svojstava iz grafa znanja EU Portala otvorenih podataka sa sveobuhvatnim statistikama korištenja. Ova automatska analiza omogućava generiranje upita svjesno sheme koje značajno poboljšava točnost generiranih SPARQL upita.

\begin{lstlisting}[language=Python, caption=Primjer testnih rezultata]
# Rezultati testiranja na 100 upita
test_results = {
    "total_queries": 100,
    "successful_queries": 92,
    "failed_queries": 8,
    "success_rate": 0.92,
    "average_response_time": 8.3,
    "vector_search_time": 0.8,
    "sparql_generation_time": 3.2,
    "schema_extraction_time": 1.5
}

# Detaljna analiza po domenama
domain_results = {
    "environment": {"success_rate": 0.95, "avg_time": 7.8},
    "energy": {"success_rate": 0.88, "avg_time": 9.1},
    "health": {"success_rate": 0.90, "avg_time": 8.5},
    "transport": {"success_rate": 0.93, "avg_time": 7.9},
    "economics": {"success_rate": 0.89, "avg_time": 8.7}
}
\end{lstlisting}

Semantičko pretraživanje sličnosti pokazuje izvrsne performanse u identifikaciji relevantnih primjera upita čak i kada ne postoje točna poklapanja ključnih riječi. Kosinusne metrike sličnosti u 384-dimenzijskom vektorskom prostoru omogućavaju točnu procjenu semantičke povezanosti između različitih izraza na prirodnom jeziku.

Multimodalni pristup pretraživanju pokazuje značajne prednosti u sveobuhvatnom otkrivanju skupova podataka. Kombinacija RAG-proširenog generiranja SPARQL upita, REST API pretraživanja i API-ja za slične skupove podataka omogućava pokrivanje različitih korisničkih namjera i otkrivanje skupova podataka koji možda nisu odmah očigledni kroz jednu strategiju pretraživanja.

\section{Usporedna analiza i robusnost sustava}

Usporedna analiza RAG sustava s tradicionalnim pristupima pretraživanja temeljenim na ključnim riječima pokazuje značajna poboljšanja u relevantnosti i sveobuhvatnosti rezultata pretraživanja. Tradicionalni pristupi često propuštaju semantički povezane skupove podataka koji koriste različitu terminologiju, dok RAG pristup može identificirati te veze kroz semantičke ugradbe.

Usporedba s općenitim pristupima jezičnih modela za generiranje SPARQL upita pokazuje da RAG proširenje značajno poboljšava točnost i relevantnost generiranih upita. Kontekst pružen kroz dohvaćene primjere i informacije o shemi omogućava modelima bolje razumijevanje ciljne domene i generiranje prikladnijih upita.

\begin{lstlisting}[language=Python, caption=Usporedba različitih pristupa]
comparison_results = {
    "traditional_keyword_search": {
        "success_rate": 0.65,
        "avg_time": 2.1,
        "semantic_accuracy": 0.58
    },
    "general_llm_approach": {
        "success_rate": 0.78,
        "avg_time": 5.2,
        "semantic_accuracy": 0.72
    },
    "rag_enhanced_approach": {
        "success_rate": 0.92,
        "avg_time": 8.3,
        "semantic_accuracy": 0.89
    }
}
\end{lstlisting}

Usporedba performansi s postojećim alatima za otkrivanje otvorenih podataka pokazuje da RAG sustav nudi jedinstvene mogućnosti u obradi upita na prirodnom jeziku i semantičkom otkrivanju skupova podataka. Dok postojeći alati mogu ponuditi brža vremena odziva za jednostavna pretraživanja ključnih riječi, RAG pristup pruža superiorne rezultate za složene analitičke upite.

Evaluacija robusnosti sustava fokusira se na procjenu ponašanja sustava pod različitim stresnim uvjetima i scenarijima grešaka. Testiranje pokazuje da sustav održava stabilno funkcioniranje čak i kada pojedinačne komponente doživljavaju privremene kvarove ili degradaciju performansi.

Mehanizmi rukovanja greškama uspješno upravljaju različitim scenarijima kvarova uključujući vremenska ograničenja mreže, ograničenja brzine API-ja i pogrešne korisničke unose. Strategije gracioznog pada omogućavaju nastavak rada s ograničenom funkcionalnošću umjesto potpunog kvara sustava.

\begin{lstlisting}[language=Python, caption=Testiranje robusnosti]
robustness_tests = {
    "network_timeout": {
        "test_count": 20,
        "successful_fallbacks": 18,
        "avg_recovery_time": 2.3
    },
    "api_rate_limit": {
        "test_count": 15,
        "successful_fallbacks": 14,
        "avg_recovery_time": 1.8
    },
    "invalid_user_input": {
        "test_count": 25,
        "successful_handling": 24,
        "avg_error_response_time": 0.5
    },
    "system_overload": {
        "test_count": 10,
        "successful_degradation": 9,
        "avg_response_time_under_load": 12.5
    }
}
\end{lstlisting}

Testiranje opterećenja pokazuje da sustav može podnijeti više istovremenih korisnika bez značajne degradacije performansi. Asinkrone mogućnosti obrade i strategije predmemoriranja omogućavaju učinkovito korištenje resursa i dosljedno korisničko iskustvo čak i pod visokim uvjetima opterećenja.

Testiranje oporavka pokazuje da se sustav može uspješno ponovno pokrenuti i nastaviti normalno funkcioniranje nakon kvarova sustava. ChromaDB trajno pohranjivanje osigurava da se vektorske ugradbe i predmemorirane informacije čuvaju kroz ponovne pokretanja sustava, omogućavajući brza vremena oporavka.

\section{Ograničenja, problemi i smjernice za budući rad}

Trenutna ograničenja RAG sustava pružaju jasne smjerove za buduće istraživanje i razvojne napore. Ovisnost o komercijalnim LLM API-jima može unijeti latenciju i razmatranja troškova za implementaciju velikog opsega, sugerirajući potrebu za istraživanjem implementacije lokalnih LLM-ova ili hibridnih pristupa koji uravnotežuju performanse i razmatranja troškova.

Podrška za jezike trenutno je prvenstveno fokusirana na engleski sadržaj, iako arhitektura sustava omogućava proširivanje za višejezičnu podršku kroz odgovarajuće modele ugradbi i podatke za treniranje. Buduća istraživanja mogu istražiti specijalizirane modele za hrvatski i druge europske jezike, omogućavajući širu dostupnost različitim korisničkim zajednicama.

Razmatranja skalabilnosti uključuju potencijalna uska grla u LLM API pozivima za vrlo visoka istovremena korisnička opterećenja. Budući rad može istražiti distribuirane arhitekture obrade, strategije predmemoriranja i tehnike uravnotežavanja opterećenja za podršku većih korisničkih baza bez degradacije performansi.

Domensko usmjeravanje trenutno je optimizirano za EU Portal otvorenih podataka, iako arhitektura sustava omogućava prilagodbu drugim grafovima znanja i portalima podataka. Buduća istraživanja mogu istražiti tehnike generalizacije za širu primjenjivost kroz različite domene i izvore podataka, omogućavajući šire usvajanje RAG pristupa.

Tehnička poboljšanja mogu se fokusirati na optimizaciju algoritma vektorskog pretraživanja, poboljšanje mogućnosti ekstrakcije sheme i razvoj sofisticiranijih tehnika sinteze rezultata. Napredne mogućnosti vizualizacije i kolaborativne značajke također predstavljaju obećavajuće smjerove za budući razvoj.

Evaluacija kvalitete metapodataka provedena je prema pristupu iz~\cite{neumaier2016automated}. 