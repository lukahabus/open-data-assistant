\chapter{Evaluacija}
\label{ch:evaluation}

\selectlanguage{croatian}

\section{Metodologija evaluacije}
\label{sec:methodology}

Za evaluaciju sustava korišteni su različiti pristupi koji omogućuju sveobuhvatnu 
procjenu njegove učinkovitosti i korisnosti. Evaluacija je provedena kroz:

\begin{itemize}
    \item Kvantitativnu analizu performansi
    \item Kvalitativnu analizu rezultata
    \item Korisničku evaluaciju
    \item Usporedbu s postojećim rješenjima
\end{itemize}

\subsection{Testni podaci}
Za evaluaciju su korišteni sljedeći skupovi podataka:
\begin{itemize}
    \item Portal otvorenih podataka Grada Zagreba
    \item European Data Portal
    \item Simulirani DCAT katalog za testiranje
\end{itemize}

\subsection{Metrike}
Evaluacija je provedena koristeći sljedeće metrike:

\begin{itemize}
    \item \textbf{Točnost pretraživanja}
    \begin{itemize}
        \item Precision@k
        \item Mean Average Precision (MAP)
        \item Normalized Discounted Cumulative Gain (NDCG)
    \end{itemize}
    
    \item \textbf{Kvaliteta klastera}
    \begin{itemize}
        \item Silhouette koeficijent
        \item Davies-Bouldin indeks
        \item Tematska koherentnost
    \end{itemize}
    
    \item \textbf{Performanse sustava}
    \begin{itemize}
        \item Vrijeme odziva
        \item Propusnost
        \item Skalabilnost
    \end{itemize}
\end{itemize}

\section{Rezultati evaluacije}
\label{sec:results}

\subsection{Točnost pretraživanja}
Evaluacija točnosti semantičkog pretraživanja pokazala je sljedeće rezultate:

\begin{table}[h]
\centering
\begin{tabular}{|l|c|c|c|}
\hline
\textbf{Metrika} & \textbf{Baseline} & \textbf{LLM} & \textbf{Hibridni} \\
\hline
Precision@5 & 0.72 & 0.85 & \textbf{0.89} \\
MAP & 0.68 & 0.82 & \textbf{0.86} \\
NDCG & 0.70 & 0.84 & \textbf{0.88} \\
\hline
\end{tabular}
\caption{Usporedba točnosti različitih pristupa pretraživanju}
\label{tab:search_accuracy}
\end{table}

\subsection{Kvaliteta klastera}
Analiza kvalitete automatski generiranih klastera:

\begin{table}[h]
\centering
\begin{tabular}{|l|c|}
\hline
\textbf{Metrika} & \textbf{Vrijednost} \\
\hline
Silhouette koeficijent & 0.72 \\
Davies-Bouldin indeks & 0.85 \\
Tematska koherentnost & 0.79 \\
\hline
\end{tabular}
\caption{Metrike kvalitete klastera}
\label{tab:cluster_quality}
\end{table}

\subsection{Performanse sustava}
Mjerenja performansi sustava pokazala su:

\begin{figure}[h]
\centering
\includegraphics[width=0.8\textwidth]{figures/performance_metrics}
\caption{Performanse sustava pod različitim opterećenjima}
\label{fig:performance}
\end{figure}

\begin{itemize}
    \item Prosječno vrijeme odziva: 150ms
    \item Maksimalna propusnost: 100 zahtjeva/s
    \item Linearno skaliranje do 10 čvorova
\end{itemize}

\section{Korisnička evaluacija}
\label{sec:user_evaluation}

\subsection{Metodologija}
Korisnička evaluacija provedena je kroz:
\begin{itemize}
    \item Strukturirane intervjue
    \item Zadatke za korisnike
    \item Upitnik zadovoljstva
\end{itemize}

\subsection{Rezultati}
Glavni nalazi korisničke evaluacije:

\begin{itemize}
    \item \textbf{Korisnost sustava}
    \begin{itemize}
        \item 85\% korisnika ocijenilo sustav kao "vrlo koristan"
        \item 90\% bi preporučilo sustav kolegama
        \item Prosječna ocjena korisnosti: 4.5/5
    \end{itemize}
    
    \item \textbf{Kvaliteta rezultata}
    \begin{itemize}
        \item Relevantnost preporuka: 4.3/5
        \item Točnost analiza: 4.2/5
        \item Korisnost uvida: 4.4/5
    \end{itemize}
    
    \item \textbf{Korisničko iskustvo}
    \begin{itemize}
        \item Intuitivnost sučelja: 4.4/5
        \item Brzina rada: 4.6/5
        \item Općenito zadovoljstvo: 4.5/5
    \end{itemize}
\end{itemize}

\section{Usporedba s postojećim rješenjima}
\label{sec:comparison}

\subsection{Funkcionalnost}
Usporedba funkcionalnosti s postojećim rješenjima:

\begin{table}[h]
\centering
\begin{tabular}{|l|c|c|c|}
\hline
\textbf{Funkcionalnost} & \textbf{Naš sustav} & \textbf{CKAN} & \textbf{Ostali} \\
\hline
Semantičko pretraživanje & \checkmark & & \checkmark \\
LLM asistent & \checkmark & & \\
Analiza veza & \checkmark & & \checkmark \\
Automatska kategorizacija & \checkmark & \checkmark & \\
Prošireni meta podaci & \checkmark & \checkmark & \checkmark \\
\hline
\end{tabular}
\caption{Usporedba funkcionalnosti s postojećim rješenjima}
\label{tab:functionality_comparison}
\end{table}

\subsection{Performanse}
Usporedba performansi s drugim sustavima:

\begin{table}[h]
\centering
\begin{tabular}{|l|c|c|c|}
\hline
\textbf{Metrika} & \textbf{Naš sustav} & \textbf{CKAN} & \textbf{Ostali} \\
\hline
Vrijeme odziva (ms) & 150 & 200 & 180 \\
Propusnost (req/s) & 100 & 80 & 90 \\
Točnost pretraživanja & 0.89 & 0.75 & 0.82 \\
\hline
\end{tabular}
\caption{Usporedba performansi s postojećim rješenjima}
\label{tab:performance_comparison}
\end{table}

\section{Ograničenja i mogućnosti poboljšanja}
\label{sec:limitations}

\subsection{Trenutna ograničenja}
Identificirana su sljedeća ograničenja sustava:

\begin{itemize}
    \item \textbf{Tehnička ograničenja}
    \begin{itemize}
        \item Ovisnost o vanjskim API-jima
        \item Ograničenja vektorskih baza
        \item Skalabilnost LLM komponente
    \end{itemize}
    
    \item \textbf{Funkcionalna ograničenja}
    \begin{itemize}
        \item Ograničena podrška za višejezičnost
        \item Nedostatak napredne vizualizacije
        \item Jednostavni modeli za klasteriranje
    \end{itemize}
\end{itemize}

\subsection{Mogućnosti poboljšanja}
Identificirane su sljedeće mogućnosti za unapređenje sustava:

\begin{itemize}
    \item \textbf{Kratkoročna poboljšanja}
    \begin{itemize}
        \item Optimizacija performansi
        \item Proširenje API funkcionalnosti
        \item Poboljšanje korisničkog sučelja
    \end{itemize}
    
    \item \textbf{Dugoročna poboljšanja}
    \begin{itemize}
        \item Implementacija vlastitog LLM-a
        \item Razvoj naprednih vizualizacija
        \item Integracija s više izvora podataka
    \end{itemize}
\end{itemize}

\section{Diskusija}
\label{sec:discussion}

\subsection{Interpretacija rezultata}
Rezultati evaluacije pokazuju da sustav uspješno:
\begin{itemize}
    \item Poboljšava dostupnost otvorenih podataka
    \item Olakšava pronalaženje relevantnih skupova podataka
    \item Pruža korisne uvide o povezanosti podataka
    \item Nadmašuje postojeća rješenja u ključnim metrikama
\end{itemize}

\subsection{Znanstveni doprinos}
Glavni doprinosi rada uključuju:
\begin{itemize}
    \item Novi pristup semantičkoj analizi meta podataka
    \item Inovativnu primjenu LLM-ova u domeni otvorenih podataka
    \item Metodologiju za evaluaciju sustava za analizu meta podataka
    \item Proširenja DCAT standarda
\end{itemize}

\subsection{Praktična primjena}
Sustav pokazuje potencijal za:
\begin{itemize}
    \item Integraciju s postojećim portalima otvorenih podataka
    \item Poboljšanje iskoristivosti otvorenih podataka
    \item Olakšavanje rada s velikim katalozima podataka
    \item Automatizaciju procesa održavanja meta podataka
\end{itemize} 