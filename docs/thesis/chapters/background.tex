\chapter{Teorijska podloga}
\label{ch:background}

\selectlanguage{croatian}

\section{Otvoreni podaci}
\label{sec:open_data}

Otvoreni podaci predstavljaju koncept prema kojem određeni podaci trebaju biti slobodno 
dostupni svima za korištenje i ponovno korištenje bez ograničenja \cite{janssen2012benefits}. 
U kontekstu javne uprave i znanstvenih institucija, otvoreni podaci igraju ključnu ulogu 
u promicanju transparentnosti, inovacija i ekonomskog razvoja.

\subsection{Karakteristike otvorenih podataka}
Prema općeprihvaćenim načelima, otvoreni podaci moraju biti:
\begin{itemize}
    \item Dostupni - podaci moraju biti dostupni u cjelini, po razumnoj cijeni reprodukcije
    \item Ponovno iskoristivi - podaci moraju biti dostupni u obliku koji omogućuje ponovno korištenje
    \item Univerzalno sudjelovanje - svi moraju moći koristiti, ponovno koristiti i redistribuirati podatke
\end{itemize}

\subsection{Izazovi u korištenju otvorenih podataka}
Unatoč rastućoj dostupnosti otvorenih podataka, njihova stvarna iskoristivost često je 
ograničena zbog nekoliko ključnih izazova:
\begin{itemize}
    \item Fragmentacija - podaci su raspršeni kroz različite portale i formate
    \item Kvaliteta meta podataka - nepotpuni ili nekonzistentni opisi podataka
    \item Povezanost - nedostatak eksplicitnih veza između povezanih skupova podataka
    \item Pristupačnost - složeni mehanizmi pristupa i nedostatak standardizacije
\end{itemize}

\section{DCAT (Data Catalog Vocabulary)}
\label{sec:dcat}

DCAT je W3C preporuka dizajnirana za olakšavanje interoperabilnosti između podatkovnih 
kataloga objavljenih na webu \cite{dcat2020}. Predstavlja standardni model za opisivanje 
skupova podataka u podatkovnim katalozima.

\subsection{Osnovni koncepti}
DCAT definira nekoliko ključnih klasa:
\begin{itemize}
    \item \texttt{dcat:Catalog} - kolekcija meta podataka o skupovima podataka
    \item \texttt{dcat:Dataset} - kolekcija podataka koju objavljuje jedan agent
    \item \texttt{dcat:Distribution} - specifična reprezentacija skupa podataka
    \item \texttt{dcat:DataService} - servis koji omogućuje pristup podacima
\end{itemize}

\subsection{Primjena u praksi}
DCAT se široko primjenjuje u:
\begin{itemize}
    \item Portalima otvorenih podataka (npr. CKAN)
    \item Znanstvenim repozitorijima
    \item Integraciji podatkovnih kataloga
\end{itemize}

\section{Veliki jezični modeli}
\label{sec:llm}

Veliki jezični modeli (Large Language Models, LLM) predstavljaju značajan napredak u 
području obrade prirodnog jezika \cite{brown2020language}. Ovi modeli, temeljeni na 
transformerskoj arhitekturi, pokazuju impresivne sposobnosti u razumijevanju i 
generiranju teksta.

\subsection{Arhitektura i principi rada}
Moderni jezični modeli temelje se na nekoliko ključnih koncepata:
\begin{itemize}
    \item Transformerska arhitektura - omogućuje paralelnu obradu teksta
    \item Mehanizam pažnje - fokusira se na relevantne dijelove ulaznog teksta
    \item Predtrening i fino podešavanje - učenje općih i specifičnih znanja
\end{itemize}

\subsection{Primjena u analizi meta podataka}
LLM-ovi donose nekoliko prednosti u kontekstu analize meta podataka:
\begin{itemize}
    \item Semantičko razumijevanje - sposobnost razumijevanja konteksta i značenja
    \item Generiranje opisa - automatsko obogaćivanje meta podataka
    \item Otkrivanje veza - prepoznavanje semantičkih odnosa između skupova podataka
\end{itemize}

\section{Semantičko pretraživanje}
\label{sec:semantic_search}

Semantičko pretraživanje nadilazi tradicionalno tekstualno podudaranje fokusirajući se 
na razumijevanje značenja i konteksta \cite{zhang2022survey}. Ova tehnologija posebno 
je relevantna za pretraživanje i povezivanje meta podataka.

\subsection{Tehnike semantičkog pretraživanja}
Moderne tehnike semantičkog pretraživanja uključuju:
\begin{itemize}
    \item Vektorske reprezentacije - pretvaranje teksta u numeričke vektore
    \item Izračun semantičke sličnosti - mjerenje bliskosti značenja
    \item Kontekstualno rangiranje - prilagodba rezultata kontekstu upita
\end{itemize}

\subsection{Primjena u otkrivanju podataka}
Semantičko pretraživanje omogućuje:
\begin{itemize}
    \item Intuitivnije pronalaženje podataka
    \item Otkrivanje skrivenih veza između skupova podataka
    \item Poboljšanu relevantnost rezultata pretraživanja
\end{itemize}

\section{Evaluacija kvalitete meta podataka}
\label{sec:metadata_quality}

Kvaliteta meta podataka ključna je za učinkovito pronalaženje i korištenje otvorenih 
podataka \cite{neumaier2016automated}. Evaluacija kvalitete obuhvaća nekoliko dimenzija.

\subsection{Dimenzije kvalitete}
Ključne dimenzije kvalitete meta podataka uključuju:
\begin{itemize}
    \item Potpunost - prisutnost svih relevantnih informacija
    \item Točnost - preciznost i istinitost informacija
    \item Konzistentnost - usklađenost s definiranim standardima
    \item Pravodobnost - ažurnost informacija
\end{itemize}

\subsection{Metrike i mjerenje}
Za evaluaciju kvalitete koriste se različite metrike:
\begin{itemize}
    \item Automatske provjere usklađenosti
    \item Semantička analiza sadržaja
    \item Korisnička povratna informacija
    \item Statističke analize kompletnosti
\end{itemize} 