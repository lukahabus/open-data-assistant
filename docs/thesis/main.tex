\documentclass[12pt,a4paper]{report}

% Language and encoding
\usepackage[utf8]{inputenc}
\usepackage[croatian,english]{babel}
\usepackage{csquotes}

% Math and symbols
\usepackage{amsmath}
\usepackage{amsfonts}
\usepackage{amssymb}

% Graphics and figures
\usepackage{graphicx}
\usepackage{float}
\usepackage{subcaption}

% Code listings
\usepackage{listings}
\usepackage{xcolor}

% Bibliography
\usepackage[backend=biber,style=ieee]{biblatex}
\addbibresource{references.bib}

% Hyperlinks
\usepackage{hyperref}
\hypersetup{
    colorlinks=true,
    linkcolor=blue,
    filecolor=magenta,
    urlcolor=cyan,
}

% Custom commands
\newcommand{\code}[1]{\texttt{#1}}
\newcommand{\system}{DCAT Metadata Analysis System}

% Document settings
\title{Sustav za analizu meta podataka otvorenih skupova podataka\\
       \large A System for Metadata Analysis of Open Datasets}
\author{Your Name}
\date{\today}

% Begin document
\begin{document}

% Front matter
\frontmatter
\maketitle

\begin{abstract}
\selectlanguage{croatian}
% Croatian abstract
Sve veća dostupnost otvorenih podataka ne podrazumijeva i njihovu veću iskoristivost. 
Iako normirani formati meta podataka (npr. DCAT) omogućuju jednostavnije pronalaženje i 
tumačenje objavljenih pojedinačnih skupova podataka, dodatna vrijednost nalazi se u 
njihovom povezivanju i pronalaženju novih uvida i skrivenog znanja. Ovaj rad predstavlja 
sustav za analizu meta podataka otvorenih skupova podataka koji koristi velike jezične 
modele za poboljšanje iskoristivosti i povezanosti otvorenih podataka.

\selectlanguage{english}
% English abstract
The increasing availability of open data does not necessarily imply its greater usability. 
While standardized metadata formats (e.g., DCAT) enable easier discovery and interpretation 
of individual published datasets, additional value lies in connecting them and discovering 
new insights and hidden knowledge. This thesis presents a system for analyzing open dataset 
metadata that utilizes large language models to improve the usability and connectivity of 
open data.
\end{abstract}

\tableofcontents
\listoffigures
\listoftables

% Main content
\mainmatter

% Include chapters
\chapter{Uvod}
\label{ch:introduction}

\selectlanguage{croatian}

\section{Motivacija}
\label{sec:motivation}

U današnje vrijeme, otvoreni podaci predstavljaju ključan resurs za inovacije, 
transparentnost i razvoj digitalnog društva. Javne institucije, istraživačke organizacije 
i privatni sektor objavljuju sve veće količine otvorenih podataka. Međutim, sama 
dostupnost podataka ne jamči njihovu iskoristivost. Jedan od ključnih izazova je 
učinkovito pronalaženje, razumijevanje i povezivanje različitih skupova podataka.

Normirani formati meta podataka, poput DCAT-a (Data Catalog Vocabulary), omogućuju 
standardizirani opis skupova podataka. No, tradicionalni pristupi pretraživanju i analizi 
meta podataka često su ograničeni na jednostavno tekstualno podudaranje ili preddefinirane 
kategorije. Ovo ograničava mogućnosti otkrivanja skrivenih veza između skupova podataka 
i otežava korisnicima pronalaženje relevantnih informacija.

\section{Ciljevi rada}
\label{sec:objectives}

Glavni cilj ovog rada je razvoj sustava koji će unaprijediti iskoristivost otvorenih 
podataka kroz naprednu analizu njihovih meta podataka. Specifični ciljevi uključuju:

\begin{itemize}
    \item Istraživanje mogućnosti velikih jezičnih modela u kontekstu analize meta podataka
    \item Razvoj sustava za semantičku analizu i povezivanje skupova podataka
    \item Implementaciju intuitivnog korisničkog sučelja za interakciju s meta podacima
    \item Integraciju s postojećim portalima otvorenih podataka (CKAN)
    \item Evaluaciju učinkovitosti i korisnosti predloženog rješenja
\end{itemize}

\section{Doprinosi}
\label{sec:contributions}

Glavni doprinosi ovog rada su:

\begin{enumerate}
    \item Nova metodologija za semantičku analizu meta podataka korištenjem velikih jezičnih modela
    \item Prošireni DCAT model s podrškom za semantičke veze između skupova podataka
    \item Implementacija sustava koji demonstrira praktičnu primjenu predložene metodologije
    \item Empirijska evaluacija učinkovitosti sustava u stvarnim uvjetima
\end{enumerate}

\section{Struktura rada}
\label{sec:structure}

Rad je organiziran na sljedeći način:

\begin{itemize}
    \item Poglavlje \ref{ch:background} predstavlja teorijsku podlogu rada, uključujući 
    pregled otvorenih podataka, DCAT standarda i velikih jezičnih modela.
    
    \item Poglavlje \ref{ch:related_work} daje pregled postojećih rješenja i 
    istraživanja u području analize meta podataka.
    
    \item Poglavlje \ref{ch:system_design} opisuje arhitekturu i dizajn predloženog 
    sustava.
    
    \item Poglavlje \ref{ch:implementation} detaljno predstavlja implementaciju sustava, 
    uključujući ključne komponente i tehnička rješenja.
    
    \item Poglavlje \ref{ch:evaluation} prikazuje rezultate evaluacije sustava i 
    diskusiju o njegovoj učinkovitosti.
    
    \item Poglavlje \ref{ch:conclusion} donosi zaključke rada i smjernice za buduća 
    istraživanja.
\end{itemize}

\selectlanguage{english}

% English version of the introduction
\chapter*{Introduction}
\addcontentsline{toc}{chapter}{Introduction (English)}

\section*{Motivation}
In today's world, open data represents a key resource for innovation, transparency, and 
the development of digital society. Public institutions, research organizations, and the 
private sector are publishing increasing amounts of open data. However, data availability 
alone does not guarantee its usability. One of the key challenges is effectively finding, 
understanding, and connecting different datasets.

Standardized metadata formats, such as DCAT (Data Catalog Vocabulary), enable standardized 
description of datasets. However, traditional approaches to searching and analyzing metadata 
are often limited to simple text matching or predefined categories. This limits the 
possibilities of discovering hidden connections between datasets and makes it difficult 
for users to find relevant information.

\section*{Objectives}
The main objective of this thesis is to develop a system that will enhance the usability 
of open data through advanced analysis of their metadata. Specific objectives include:

\begin{itemize}
    \item Exploring the capabilities of large language models in the context of metadata analysis
    \item Developing a system for semantic analysis and linking of datasets
    \item Implementing an intuitive user interface for metadata interaction
    \item Integration with existing open data portals (CKAN)
    \item Evaluating the effectiveness and usefulness of the proposed solution
\end{itemize}

\section*{Contributions}
The main contributions of this work are:

\begin{enumerate}
    \item A new methodology for semantic metadata analysis using large language models
    \item Extended DCAT model with support for semantic relationships between datasets
    \item Implementation of a system demonstrating practical application of the proposed methodology
    \item Empirical evaluation of system effectiveness in real conditions
\end{enumerate}

\section*{Structure}
The thesis is organized as follows:

\begin{itemize}
    \item Chapter \ref{ch:background} presents the theoretical background, including an 
    overview of open data, DCAT standard, and large language models.
    
    \item Chapter \ref{ch:related_work} provides an overview of existing solutions and 
    research in the field of metadata analysis.
    
    \item Chapter \ref{ch:system_design} describes the architecture and design of the 
    proposed system.
    
    \item Chapter \ref{ch:implementation} presents the system implementation in detail, 
    including key components and technical solutions.
    
    \item Chapter \ref{ch:evaluation} shows the results of system evaluation and 
    discusses its effectiveness.
    
    \item Chapter \ref{ch:conclusion} provides conclusions and directions for future 
    research.
\end{itemize} 
\chapter{Teorijska podloga}
\label{ch:background}

\selectlanguage{croatian}

\section{Otvoreni podaci}
\label{sec:open_data}

Otvoreni podaci predstavljaju koncept prema kojem određeni podaci trebaju biti slobodno 
dostupni svima za korištenje i ponovno korištenje bez ograničenja \cite{janssen2012benefits}. 
U kontekstu javne uprave i znanstvenih institucija, otvoreni podaci igraju ključnu ulogu 
u promicanju transparentnosti, inovacija i ekonomskog razvoja.

\subsection{Karakteristike otvorenih podataka}
Prema općeprihvaćenim načelima, otvoreni podaci moraju biti:
\begin{itemize}
    \item Dostupni - podaci moraju biti dostupni u cjelini, po razumnoj cijeni reprodukcije
    \item Ponovno iskoristivi - podaci moraju biti dostupni u obliku koji omogućuje ponovno korištenje
    \item Univerzalno sudjelovanje - svi moraju moći koristiti, ponovno koristiti i redistribuirati podatke
\end{itemize}

\subsection{Izazovi u korištenju otvorenih podataka}
Unatoč rastućoj dostupnosti otvorenih podataka, njihova stvarna iskoristivost često je 
ograničena zbog nekoliko ključnih izazova:
\begin{itemize}
    \item Fragmentacija - podaci su raspršeni kroz različite portale i formate
    \item Kvaliteta meta podataka - nepotpuni ili nekonzistentni opisi podataka
    \item Povezanost - nedostatak eksplicitnih veza između povezanih skupova podataka
    \item Pristupačnost - složeni mehanizmi pristupa i nedostatak standardizacije
\end{itemize}

\section{DCAT (Data Catalog Vocabulary)}
\label{sec:dcat}

DCAT je W3C preporuka dizajnirana za olakšavanje interoperabilnosti između podatkovnih 
kataloga objavljenih na webu \cite{dcat2020}. Predstavlja standardni model za opisivanje 
skupova podataka u podatkovnim katalozima.

\subsection{Osnovni koncepti}
DCAT definira nekoliko ključnih klasa:
\begin{itemize}
    \item \texttt{dcat:Catalog} - kolekcija meta podataka o skupovima podataka
    \item \texttt{dcat:Dataset} - kolekcija podataka koju objavljuje jedan agent
    \item \texttt{dcat:Distribution} - specifična reprezentacija skupa podataka
    \item \texttt{dcat:DataService} - servis koji omogućuje pristup podacima
\end{itemize}

\subsection{Primjena u praksi}
DCAT se široko primjenjuje u:
\begin{itemize}
    \item Portalima otvorenih podataka (npr. CKAN)
    \item Znanstvenim repozitorijima
    \item Integraciji podatkovnih kataloga
\end{itemize}

\section{Veliki jezični modeli}
\label{sec:llm}

Veliki jezični modeli (Large Language Models, LLM) predstavljaju značajan napredak u 
području obrade prirodnog jezika \cite{brown2020language}. Ovi modeli, temeljeni na 
transformerskoj arhitekturi, pokazuju impresivne sposobnosti u razumijevanju i 
generiranju teksta.

\subsection{Arhitektura i principi rada}
Moderni jezični modeli temelje se na nekoliko ključnih koncepata:
\begin{itemize}
    \item Transformerska arhitektura - omogućuje paralelnu obradu teksta
    \item Mehanizam pažnje - fokusira se na relevantne dijelove ulaznog teksta
    \item Predtrening i fino podešavanje - učenje općih i specifičnih znanja
\end{itemize}

\subsection{Primjena u analizi meta podataka}
LLM-ovi donose nekoliko prednosti u kontekstu analize meta podataka:
\begin{itemize}
    \item Semantičko razumijevanje - sposobnost razumijevanja konteksta i značenja
    \item Generiranje opisa - automatsko obogaćivanje meta podataka
    \item Otkrivanje veza - prepoznavanje semantičkih odnosa između skupova podataka
\end{itemize}

\section{Semantičko pretraživanje}
\label{sec:semantic_search}

Semantičko pretraživanje nadilazi tradicionalno tekstualno podudaranje fokusirajući se 
na razumijevanje značenja i konteksta \cite{zhang2022survey}. Ova tehnologija posebno 
je relevantna za pretraživanje i povezivanje meta podataka.

\subsection{Tehnike semantičkog pretraživanja}
Moderne tehnike semantičkog pretraživanja uključuju:
\begin{itemize}
    \item Vektorske reprezentacije - pretvaranje teksta u numeričke vektore
    \item Izračun semantičke sličnosti - mjerenje bliskosti značenja
    \item Kontekstualno rangiranje - prilagodba rezultata kontekstu upita
\end{itemize}

\subsection{Primjena u otkrivanju podataka}
Semantičko pretraživanje omogućuje:
\begin{itemize}
    \item Intuitivnije pronalaženje podataka
    \item Otkrivanje skrivenih veza između skupova podataka
    \item Poboljšanu relevantnost rezultata pretraživanja
\end{itemize}

\section{Evaluacija kvalitete meta podataka}
\label{sec:metadata_quality}

Kvaliteta meta podataka ključna je za učinkovito pronalaženje i korištenje otvorenih 
podataka \cite{neumaier2016automated}. Evaluacija kvalitete obuhvaća nekoliko dimenzija.

\subsection{Dimenzije kvalitete}
Ključne dimenzije kvalitete meta podataka uključuju:
\begin{itemize}
    \item Potpunost - prisutnost svih relevantnih informacija
    \item Točnost - preciznost i istinitost informacija
    \item Konzistentnost - usklađenost s definiranim standardima
    \item Pravodobnost - ažurnost informacija
\end{itemize}

\subsection{Metrike i mjerenje}
Za evaluaciju kvalitete koriste se različite metrike:
\begin{itemize}
    \item Automatske provjere usklađenosti
    \item Semantička analiza sadržaja
    \item Korisnička povratna informacija
    \item Statističke analize kompletnosti
\end{itemize} 
\include{chapters/related_work}
\chapter{Dizajn sustava}
\label{ch:system_design}

\selectlanguage{croatian}

\section{Arhitektura sustava}
\label{sec:architecture}

Predloženi RAG sustav temelji se na troslojnoj arhitekturi koja omogućuje skalabilnu i održivu implementaciju naprednog semantičkog pretraživanja nad EU Portalom otvorenih podataka. Arhitektura je dizajnirana prema načelima modularnosti, proširivosti i optimizacije performansi, omogućujući robusno funkcioniranje u produkcijskom okruženju.

\begin{figure}[htbp]
    \centering
    \includegraphics[width=\textwidth]{figures/system_architecture.png}
    \caption{Arhitektura sustava za analizu DCAT metapodataka}
    \label{fig:system_architecture}
\end{figure}

Sloj pohrane čini ChromaDB vektorska baza podataka koja omogućuje trajno pohranjivanje visokodimenzijskih ugradbi s optimiziranim mogućnostima pretraživanja sličnosti. Ovaj sloj također uključuje predmemoriju informacija o shemi i mehanizme predmemoriranja rezultata upita koji omogućuju značajna poboljšanja performansi za ponavljane operacije.

Sloj obrade sastoji se od Sentence Transformers modela za generiranje ugradbi, OpenAI GPT-4 integracije za generiranje SPARQL upita, te komponenti za automatsku ekstrakciju sheme koje dinamički analiziraju strukturu grafa znanja. Ovaj sloj implementira osnovnu RAG funkcionalnost kroz inteligentno dohvaćanje i procese proširenog generiranja.

Sloj sučelja omogućuje multimodalnu interakciju kroz LangChain agent okvir koji orkestrira različite strategije pretraživanja. Arhitektura temeljena na agentima omogućuje sofisticirano upravljanje tijekom rada s automatskim strategijama za vraćanje i inteligentnu sintezu rezultata kroz više izvora podataka.

\section{Ključne komponente}
\label{sec:key_components}

Sustav se sastoji od sljedećih glavnih komponenti koje implementiraju sveobuhvatnu RAG funkcionalnost. RAG System komponenta implementira osnovnu logiku povratnog dohvaćanja i generiranja kroz ChromaDB integraciju za pohranu vektora, Sentence Transformers za generiranje ugradbi, i OpenAI GPT-4 za generiranje upita. Schema Extractor komponenta automatski analizira strukturu grafa znanja EU Portala otvorenih podataka kroz ekstrakciju VoID opisa i DCAT-specifičnu analizu. Unified Data Assistant komponenta orkestrira multimodalne strategije pretraživanja kroz LangChain agent okvir koji kombinira SPARQL upite, REST API pozive i funkcionalnost API-ja za slične skupove podataka. Vektorska baza podataka komponenta omogućuje trajno pohranjivanje i učinkovito dohvaćanje semantičkih ugradbi kroz optimizirane algoritme pretraživanja sličnosti. Komponenta za validaciju upita implementira dvostupanjski proces validacije koji osigurava sintaksnu ispravnost i izvodljivost generiranih SPARQL upita. Komponenta za optimizaciju performansi implementira strategije predmemoriranja, upravljanje ograničenjima tokena i asinkrone mogućnosti obrade za optimalne performanse sustava.

\section{RAG komponente i tijek rada}
\label{sec:rag_components}

RAG sustav implementira četiri ključne komponente identificirane u istraživačkoj literaturi \cite{lewis2020retrieval, reimers2019sentence, wang2023vector}. Svaka komponenta je dizajnirana za optimalne performanse i robusno rukovanje greškama u produkcijskom okruženju.

Komponenta za ugradbe i indeksiranje koristi all-MiniLM-L6-v2 Sentence Transformers model za generiranje 384-dimenzijskih semantičkih ugradbi iz upita na prirodnom jeziku i pohranjenih primjera. ChromaDB omogućuje učinkovito pohranjivanje i dohvaćanje ovih ugradbi kroz optimizirane algoritme pretraživanja sličnosti koji koriste kosinusne metrike udaljenosti.

Komponenta za izgradnju promptova implementira sofisticiran proces sastavljanja konteksta koji kombinira dohvaćene slične primjere s relevantnim informacijama o shemi. Ovaj proces omogućuje stvaranje sveobuhvatnih promptova koji pružaju dovoljan kontekst za točno generiranje SPARQL upita dok ostaju unutar ograničenja tokena komercijalnih LLM API-ja.

Komponenta za validaciju upita implementira dvostupanjski proces validacije koji uključuje provjeru sintakse kroz parsiranje SPARQL-a i validaciju izvršavanja kroz testne upite s ograničenjima LIMIT 1. Ovaj pristup omogućuje rano otkrivanje sintaksnih grešaka i provjeru da se generirani upiti mogu uspješno izvršiti protiv ciljne krajnje točke.

\section{Multimodalni pristup pretraživanju}
\label{sec:multimodal_approach}

Unified Data Assistant implementira multimodalni pristup koji kombinira tri komplementarne strategije pretraživanja za sveobuhvatno otkrivanje skupova podataka. Ovaj pristup omogućuje optimalno pokrivanje različitih korisničkih namjera i tipova upita kroz inteligentnu orkestraciju više metoda pristupa podacima.

RAG-prošireno generiranje SPARQL upita predstavlja primarnu strategiju pretraživanja koja koristi semantičko pretraživanje sličnosti za dohvaćanje relevantnih primjera upita i informacija o shemi. Ovaj pristup je optimiziran za strukturirane upite koji zahtijevaju precizno filtriranje i složena spajanja kroz više svojstava skupova podataka.

REST API pretraživanje omogućuje fleksibilno pretraživanje temeljeno na ključnim riječima s mogućnostima facetiranja preko izdavača, formata, teme i vremenskih dimenzija. Ovaj pristup je idealan za istraživačke upite gdje korisnici nisu sigurni o točnoj terminologiji ili strukturi ciljnih skupova podataka.

API za slične skupove podataka koristi vlastite algoritme platforme sličnosti za otkrivanje povezanih skupova podataka na temelju sličnosti metapodataka. Ovaj pristup omogućuje neočekivano otkrivanje povezanih skupova podataka koji možda nisu odmah očigledni kroz tradicionalne metode pretraživanja.

\section{Ekstrakcija sheme i analiza grafa znanja}
\label{sec:schema_extraction}

Sustav za automatsku ekstrakciju sheme implementira sveobuhvatnu analizu strukture grafa znanja EU Portala otvorenih podataka kroz kombinaciju ekstrakcije VoID opisa i DCAT-specifične analize. Ovaj sustav omogućuje dinamičko razumijevanje strukture skupa podataka i uzoraka korištenja rječnika koji su bitni za informirano generiranje upita.

Komponenta za ekstrakciju VoID opisa automatski dohvaća sveobuhvatne statistike o strukturi grafa znanja uključujući ukupan broj trojki, različitih subjekata, klasa i svojstava. Ove informacije omogućuju razumijevanje opsega i složenosti ciljnog grafa znanja što je ključno za optimizaciju strategija generiranja upita.

DCAT-specifična analiza fokusira se na ekstrakciju domenski specifičnih statistika relevantnih za ekosustav EU Portala otvorenih podataka. Analiza uključuje nabrajanje broja skupova podataka, formata distribucije, statistika izdavača, pokrivanje tema i vremenske uzorke koji omogućuju specijaliziranu optimizaciju za kontekst europskih otvorenih podataka.

Nabrajanje klasa i svojstava sa statistikama korištenja omogućuje identifikaciju najčešće korištenih pojmova rječnika i njihovih odnosa unutar grafa znanja. Ova informacija je integrirana u proces izgradnje RAG promptova za generiranje upita koji slijede ustaljene uzorke i koriste odgovarajuće pojmove rječnika.

\section{Orkestracija temeljena na agentima}
\label{sec:agent_orchestration}

LangChain agent okvir omogućuje sofisticiranu orkestraciju više strategija pretraživanja kroz inteligentno upravljanje tijekom rada. Arhitektura temeljena na agentima implementira logiku donošenja odluka koja određuje optimalnu strategiju pretraživanja na temelju karakteristika upita, korisničke namjere i dostupnih resursa.

Inženjerstvo agent promptova implementira sveobuhvatne upute koje vode ponašanje agenta kroz složene višestupanjske tokove rada. Promptovi su dizajnirani da potiču sistemski pristup koji počinje s RAG-proširenim generiranjem SPARQL upita, nastavlja kroz API pretraživanja, i završava s inteligentnom sintezom i analizom rezultata.

Integracija alata omogućuje besprijekorna interakcija između različitih komponenti pretraživanja kroz standardizirana sučelja. Svaki alat implementira robusno rukovanje greškama i pruža strukturirani izlaz koji može biti lako obrađen od strane logike donošenja odluka agenta.

Komponenta za sintezu rezultata implementira inteligentnu analizu i kombinaciju rezultata iz više strategija pretraživanja. Ovaj proces uključuje uklanjanje duplikata, rangiranje relevantnosti, i sveobuhvatno sažimanje koje korisnicima pruža praktične uvide o dostupnim skupovima podataka.

\section{Strategije optimizacije performansi}
\label{sec:performance_optimization}

Optimizacija performansi u RAG sustavu implementirana je kroz predmemoriranje na više razina, inteligentno upravljanje resursima i optimizirane strukture podataka. Ove strategije omogućuju vremena odgovora u sekundama za pretraživanja sličnosti i razumna vremena odgovora za složene multimodalne upite.

Optimizacija vektorskog pretraživanja implementirana je kroz ChromaDB trajno pohranjivanje koje omogućuje brzo pokretanje sustava i dosljedne performanse kroz sesije. Operacije u skupinama i optimizirane strategije indeksiranja omogućuju učinkovito rukovanje velikim kolekcijama primjera upita bez značajne degradacije performansi.

Upravljanje ograničenjima tokena implementirano je kroz inteligentne strategije skraćivanja koje čuvaju najvažnije informacije dok ostaju unutar ograničenja komercijalnih API-ja. Tehnike sažimanja rezultata i kompresije konteksta omogućuju učinkovito rukovanje velikim skupovima rezultata bez gubitka kritičnih informacija.

Asinkrone mogućnosti obrade omogućuju paralelno izvršavanje više strategija pretraživanja što rezultira bržim ukupnim vremenima odgovora. Strategije predmemoriranja na više razina uključujući predmemoriju ugradbi, predmemoriju sheme i predmemoriju rezultata omogućuju značajna poboljšanja performansi za ponavljane upite.

\section{Rukovanje greškama i robusnost}
\label{sec:error_handling}

Sveobuhvatno rukovanje greškama implementirano je kroz više slojeva sustava za osiguravanje robusnog funkcioniranja u produkcijskom okruženju. Strategije rukovanja greškama uključuju graciozan pad, automatske mehanizme ponovnog pokušaja i sveobuhvatno bilježenje za otklanjanje grešaka i nadzor.

Validacija SPARQL upita implementira dvostupanjski pristup koji hvata sintaksne greške prije izvršavanja i pruža značajne poruke o greškama za otklanjanje grešaka. Automatski mehanizmi ponovnog pokušaja s profinjavanjem konteksta omogućuju oporavak od privremenih kvarova i poboljšanje kvalitete upita kroz iterativno pročišćavanje.

Rukovanje API greškama implementira robusne strategije za rukovanje ograničenjima brzine, vremenskim ograničenjima mreže i nedostupnosti servisa. Rezervni mehanizmi omogućuju nastavak rada čak i kada pojedinačne komponente nisu dostupne, osiguravajući dosljedno korisničko iskustvo.

Bilježenje i nadzor implementirani su kroz sveobuhvatan okvir koji prati metrike performansi, stope grešaka i uzorke korisničkih interakcija. Ove informacije omogućuju kontinuirano poboljšanje sustava i proaktivnu identifikaciju potencijalnih problema prije nego što utječu na korisničko iskustvo.

\section{Skalabilnost i razmatranja za implementaciju}
\label{sec:scalability}

Sustav je dizajniran za horizontalno skaliranje kroz modularnu arhitekturu i dizajn komponenti bez stanja. ChromaDB trajno pohranjivanje omogućuje jednostavne strategije sigurnosnog kopiranja i migracije, dok kontejnerizirano implementiranje omogućuje fleksibilnu dodjelu resursa i skaliranje na temelju potražnje.

Upravljanje resursima implementirano je kroz inteligentne strategije dodjele koje uravnotežuju točnost i performanse na temelju dostupnih računalnih resursa. Prilagodljivi algoritmi omogućuju automatsko prilagođavanje parametara obrade na temelju opterećenja sustava i zahtjeva za vremenom odgovora.

Upravljanje konfiguracije omogućuje jednostavno prilagođavanje sustava za različita okruženja implementacije i slučajeve korištenja. Parametrizirane komponente omogućuju fino podešavanje karakteristika performansi bez potrebe za izmjenama koda, omogućujući optimalno funkcioniranje u različitim produkcijskim scenarijima. 
\chapter{Implementacija}
\label{ch:implementation}

\selectlanguage{croatian}

\section{Tehnologije i alati}
\label{sec:technologies}

Za implementaciju sustava odabrane su moderne tehnologije koje omogućuju razvoj 
skalabilnog i održivog rješenja. Odabir tehnologija vođen je sljedećim kriterijima:
\begin{itemize}
    \item Zrelost i stabilnost tehnologije
    \item Dostupnost dokumentacije i zajednice
    \item Performanse i skalabilnost
    \item Jednostavnost integracije
\end{itemize}

\subsection{Backend tehnologije}
\begin{itemize}
    \item \textbf{Python} - glavni programski jezik
    \begin{itemize}
        \item FastAPI za REST API
        \item Pydantic za validaciju podataka
        \item SQLAlchemy za rad s bazom podataka
    \end{itemize}
    
    \item \textbf{Vektorska baza} - Faiss/Milvus
    \begin{itemize}
        \item Pohrana i pretraživanje embeddings
        \item Visoke performanse za similarity search
        \item Skalabilnost za velike količine podataka
    \end{itemize}
    
    \item \textbf{LLM integracija}
    \begin{itemize}
        \item OpenAI API
        \item LangChain za orkestraciju
        \item Sentence Transformers za embeddings
    \end{itemize}
\end{itemize}

\subsection{Frontend tehnologije}
\begin{itemize}
    \item \textbf{React} - biblioteka za korisničko sučelje
    \begin{itemize}
        \item TypeScript za type safety
        \item Material-UI za komponente
        \item React Query za state management
    \end{itemize}
    
    \item \textbf{Vizualizacija}
    \begin{itemize}
        \item D3.js za grafove
        \item Chart.js za statistike
        \item React Flow za interaktivne dijagrame
    \end{itemize}
\end{itemize}

\section{Implementacija komponenti}
\label{sec:components}

\subsection{DCAT Adapter}
Implementacija DCAT adaptera uključuje:
\begin{itemize}
    \item Parsiranje različitih formata meta podataka
    \item Normalizaciju u standardni format
    \item Validaciju prema DCAT shemi
    \item Proširenja za dodatne metrike
\end{itemize}

\begin{lstlisting}[language=Python, caption=Implementacija DCAT adaptera]
class DCATAdapter:
    def __init__(self):
        self.validator = DCATValidator()
        self.normalizer = MetadataNormalizer()
    
    def process_metadata(self, metadata: dict) -> DCATDataset:
        normalized = self.normalizer.normalize(metadata)
        validated = self.validator.validate(normalized)
        return DCATDataset.from_dict(validated)
\end{lstlisting}

\subsection{Embedding Engine}
Modul za generiranje i upravljanje vektorskim reprezentacijama:
\begin{itemize}
    \item Korištenje predtreniranih modela
    \item Prilagodba za specifične domene
    \item Optimizacija performansi
    \item Cachiranje rezultata
\end{itemize}

\begin{lstlisting}[language=Python, caption=Implementacija Embedding Engine-a]
class EmbeddingEngine:
    def __init__(self, model_name: str = "all-MiniLM-L6-v2"):
        self.model = SentenceTransformer(model_name)
        self.cache = EmbeddingCache()
    
    def generate_embeddings(self, text: str) -> np.ndarray:
        if cached := self.cache.get(text):
            return cached
        embedding = self.model.encode(text)
        self.cache.store(text, embedding)
        return embedding
\end{lstlisting}

\subsection{Semantic Analyzer}
Komponenta za semantičku analizu implementira:
\begin{itemize}
    \item Analizu sličnosti skupova podataka
    \item Otkrivanje tematskih klastera
    \item Generiranje semantičkih veza
    \item Procjenu kvalitete meta podataka
\end{itemize}

\begin{lstlisting}[language=Python, caption=Implementacija Semantic Analyzer-a]
class SemanticAnalyzer:
    def __init__(self, embedding_engine: EmbeddingEngine):
        self.embedding_engine = embedding_engine
        self.index = VectorIndex()
    
    def analyze_dataset(self, dataset: DCATDataset) -> Analysis:
        embeddings = self.generate_dataset_embeddings(dataset)
        similar = self.find_similar_datasets(embeddings)
        clusters = self.identify_clusters(embeddings)
        return Analysis(similar=similar, clusters=clusters)
\end{lstlisting}

\subsection{LLM Assistant}
Implementacija asistenta uključuje:
\begin{itemize}
    \item Integraciju s OpenAI API-jem
    \item Upravljanje kontekstom razgovora
    \item Generiranje prirodnih odgovora
    \item Analizu korisničkih upita
\end{itemize}

\begin{lstlisting}[language=Python, caption=Implementacija LLM Asistenta]
class LLMAssistant:
    def __init__(self, model: str = "gpt-3.5-turbo"):
        self.llm = ChatOpenAI(model=model)
        self.memory = ConversationBufferMemory()
    
    async def analyze_query(self, query: str) -> AssistantResponse:
        context = self.memory.get_context()
        response = await self.llm.generate_response(query, context)
        self.memory.add_interaction(query, response)
        return response
\end{lstlisting}

\section{API implementacija}
\label{sec:api}

REST API implementiran je korištenjem FastAPI okvira:

\begin{lstlisting}[language=Python, caption=Implementacija glavnih API endpointa]
@router.post("/query")
async def query_datasets(request: QueryRequest) -> AssistantResponse:
    response = await assistant.analyze_query(request.query)
    return response

@router.get("/datasets/{dataset_id}")
async def get_dataset(dataset_id: str) -> Dataset:
    dataset = await repository.get_dataset(dataset_id)
    return dataset

@router.post("/analyze")
async def analyze_dataset(request: AnalysisRequest) -> List[MetadataInsight]:
    insights = await analyzer.analyze_dataset(request.dataset_id)
    return insights
\end{lstlisting}

\section{Frontend implementacija}
\label{sec:frontend}

\subsection{Komponente korisničkog sučelja}
Implementirane su sljedeće React komponente:
\begin{itemize}
    \item \textbf{Dataset Explorer} - pregled i pretraživanje skupova podataka
    \item \textbf{Metadata Viewer} - detaljni prikaz meta podataka
    \item \textbf{Relationship Graph} - vizualizacija veza
    \item \textbf{Search Interface} - napredno pretraživanje
\end{itemize}

\begin{lstlisting}[language=TypeScript, caption=Implementacija Dataset komponente]
interface DatasetProps {
  dataset: Dataset;
  onAnalyze: (id: string) => void;
}

const Dataset: React.FC<DatasetProps> = ({ dataset, onAnalyze }) => {
  return (
    <Card>
      <CardHeader title={dataset.title} />
      <CardContent>
        <Typography>{dataset.description}</Typography>
        <MetadataList metadata={dataset.metadata} />
      </CardContent>
      <CardActions>
        <Button onClick={() => onAnalyze(dataset.id)}>
          Analyze
        </Button>
      </CardActions>
    </Card>
  );
};
\end{lstlisting}

\section{Testiranje}
\label{sec:testing}

\subsection{Jedinični testovi}
Implementirani su testovi za sve ključne komponente:

\begin{lstlisting}[language=Python, caption=Primjer unit testa]
def test_embedding_generation():
    engine = EmbeddingEngine()
    text = "Test dataset description"
    embedding = engine.generate_embeddings(text)
    
    assert embedding is not None
    assert embedding.shape == (384,)  # Expected dimension
    
def test_semantic_analysis():
    analyzer = SemanticAnalyzer(mock_embedding_engine)
    dataset = create_test_dataset()
    analysis = analyzer.analyze_dataset(dataset)
    
    assert len(analysis.similar) > 0
    assert len(analysis.clusters) > 0
\end{lstlisting}

\subsection{Integracijski testovi}
Testiranje integracije komponenti:

\begin{lstlisting}[language=Python, caption=Primjer integracijskog testa]
async def test_full_analysis_pipeline():
    # Setup test components
    adapter = DCATAdapter()
    engine = EmbeddingEngine()
    analyzer = SemanticAnalyzer(engine)
    
    # Test pipeline
    metadata = load_test_metadata()
    dataset = adapter.process_metadata(metadata)
    embeddings = engine.generate_embeddings(dataset.description)
    analysis = analyzer.analyze_dataset(dataset)
    
    # Verify results
    assert analysis.is_valid()
    assert len(analysis.insights) > 0
\end{lstlisting}

\section{Optimizacija}
\label{sec:optimization}

\subsection{Performanse}
Implementirane su sljedeće optimizacije:
\begin{itemize}
    \item Cachiranje embeddings
    \item Batch processing za LLM pozive
    \item Optimizacija vektorskih upita
    \item Lazy loading podataka
\end{itemize}

\subsection{Skalabilnost}
Sustav je pripremljen za skaliranje kroz:
\begin{itemize}
    \item Asinkrono procesiranje
    \item Distribuirani cache
    \item Load balancing
    \item Horizontalno skaliranje komponenti
\end{itemize} 
\chapter{Evaluacija sustava}
\label{ch:evaluation}

\selectlanguage{croatian}

Evaluacija razvijenog sustava predstavlja ključnu fazu koja omogućava objektivnu procjenu uspješnosti implementiranih rješenja. Ovo poglavlje detaljno opisuje metodologiju testiranja, testno okruženje, korištene metrike te prezentira i analizira dobivene rezultate. Evaluacija je provedena na stvarnim podacima s EU Portala otvorenih podataka, što osigurava relevantnost i primjenjivost rezultata u produkcijskom okruženju.

\section{Metodologija i testno okruženje}

Evaluacija sustava provedena je kroz sveobuhvatan pristup koji kombinira kvantitativne i kvalitativne metode analize. Cilj je bio ocijeniti ne samo tehničke performanse sustava, već i njegovu praktičnu uporabljivost za ciljane korisnike.

\subsection{Izvor podataka}

EU Portal otvorenih podataka predstavlja jedan od najvećih i najraznovrsnijih izvora otvorenih podataka u Europi. Karakteristike portala relevantne za evaluaciju:

\begin{itemize}
    \item \textbf{Broj skupova podataka}: Preko 1.5 milijuna skupova podataka iz različitih domena
    \item \textbf{Broj izdavača}: Više od 100 europskih institucija i nacionalnih portala
    \item \textbf{Jezici}: Metapodaci dostupni na 24 službena jezika EU
    \item \textbf{Formati}: Preko 200 različitih formata distribucija
    \item \textbf{SPARQL \textit{endpoint}}: \texttt{https://data.europa.eu/sparql}
    \item \textbf{Ažurnost}: Dnevno ažuriranje s novim skupovima podataka
\end{itemize}

Portal koristi DCAT-AP 2.0.1 standard za opisivanje metapodataka, što omogućava konzistentnu strukturu kroz sve skupove podataka. Ova standardizacija bila je ključna za uspješnost RAG sustava.

\subsection{Testni skup upita}

Za evaluaciju je pažljivo konstruiran testni skup od 100 upita koji pokrivaju različite domene, razine složenosti i tipove analiza. Upiti su kategorizirani na sljedeći način:

\begin{table}[htbp]
\centering
\caption{Distribucija testnih upita po domenama i složenosti}
\label{tab:test_queries_distribution}
\begin{tabular}{|l|c|c|c|c|}
\hline
\textbf{Domena} & \textbf{Jednostavni} & \textbf{Srednji} & \textbf{Složeni} & \textbf{Ukupno} \\
\hline
Okoliš & 5 & 8 & 7 & 20 \\
Energija & 4 & 6 & 5 & 15 \\
Transport & 4 & 7 & 4 & 15 \\
Zdravstvo & 3 & 5 & 7 & 15 \\
Ekonomija & 5 & 8 & 7 & 20 \\
Obrazovanje & 3 & 4 & 3 & 10 \\
Ostalo & 2 & 2 & 1 & 5 \\
\hline
\textbf{Ukupno} & 26 & 40 & 34 & 100 \\
\hline
\end{tabular}
\end{table}

Primjeri upita različite složenosti:

\begin{itemize}
    \item \textbf{Jednostavni}: "Pronađi sve skupove podataka o kvaliteti zraka"
    \item \textbf{Srednje složeni}: "Prikaži skupove podataka o potrošnji energije u Njemačkoj između 2020. i 2023. godine"
    \item \textbf{Složeni}: "Analiziraj trendove emisija CO2 u transportnom sektoru za zemlje EU-15, grupirane po godini i vrsti prijevoza"
\end{itemize}

\subsection{Definirane metrike}

Za sveobuhvatnu evaluaciju sustava definirane su sljedeće metrike:

\subsubsection{Metrike točnosti}

\begin{enumerate}
    \item \textbf{Stopa uspjeha generiranja upita (\textit{Query Generation Success Rate})}:
    $$QGSR = \frac{\text{Broj uspješno generiranih SPARQL upita}}{\text{Ukupan broj testnih upita}} \times 100\%$$
    
    \item \textbf{Sintaksna ispravnost (\textit{Syntactic Correctness})}:
    $$SC = \frac{\text{Broj sintaksno ispravnih upita}}{\text{Broj generiranih upita}} \times 100\%$$
    
    \item \textbf{Semantička točnost (\textit{Semantic Accuracy})}:
    $$SA = \frac{\text{Broj upita koji vraćaju relevantne rezultate}}{\text{Broj izvršenih upita}} \times 100\%$$
    
    \item \textbf{Preciznost rezultata (\textit{Result Precision})}:
    $$P = \frac{\text{Broj relevantnih rezultata}}{\text{Ukupan broj vraćenih rezultata}}$$
    
    \item \textbf{Odziv rezultata (\textit{Result Recall})}:
    $$R = \frac{\text{Broj pronađenih relevantnih rezultata}}{\text{Ukupan broj postojećih relevantnih rezultata}}$$
\end{enumerate}

\subsubsection{Metrike performansi}

\begin{enumerate}
    \item \textbf{Vrijeme odziva (\textit{Response Time})} - ukupno vrijeme od korisničkog upita do prikaza rezultata
    \item \textbf{Vrijeme generiranja upita (\textit{Query Generation Time})} - vrijeme potrebno za RAG proces
    \item \textbf{Vrijeme izvršavanja SPARQL upita (\textit{SPARQL Execution Time})}
    \item \textbf{Vrijeme vektorskog pretraživanja (\textit{Vector Search Time})}
    \item \textbf{Propusnost (\textit{Throughput})} - broj upita po sekundi koje sustav može obraditi
\end{enumerate}

\subsubsection{Metrike kvalitete rezultata}

\begin{enumerate}
    \item \textbf{Relevantnost rangiranja} - mjeri koliko su najrelevantniji rezultati visoko rangirani
    \item \textbf{Potpunost odgovora} - procjenjuje pokriva li odgovor sve aspekte upita
    \item \textbf{Razumljivost objašnjenja} - kvaliteta generiranog sažetka rezultata
\end{enumerate}

\subsection{Testno okruženje}

Evaluacija je provedena u kontroliranom okruženju s sljedećim specifikacijama:

\begin{itemize}
    \item \textbf{Hardver}:
    \begin{itemize}
        \item CPU: Intel Core i7-11700K @ 3.6GHz (8 jezgri, 16 threadova)
        \item RAM: 32GB DDR4 3200MHz
        \item SSD: NVMe 1TB (za ChromaDB pohranu)
        \item Mreža: 1Gbps simetrična veza
    \end{itemize}
    
    \item \textbf{Softver}:
    \begin{itemize}
        \item OS: Ubuntu 22.04 LTS
        \item Python: 3.10.12
        \item ChromaDB: 0.4.24
        \item LangChain: 0.1.16
        \item Sentence Transformers: 2.5.1
        \item OpenAI API: GPT-4 (gpt-4-0125-preview)
    \end{itemize}
    
    \item \textbf{Konfiguracija sustava}:
    \begin{itemize}
        \item Veličina vektorske baze: 5,000 primjera upita
        \item Cache veličina: 1,000 upita
        \item Paralelizam: 4 istovremena zahtjeva
        \item \textit{Timeout}: 30 sekundi po upitu
    \end{itemize}
\end{itemize}

\section{Analiza rezultata}

Rezultati evaluacije pružaju uvid u različite aspekte sustava, od tehničkih performansi do praktične uporabljivosti. Analiza je strukturirana prema definiranim kategorijama metrika.

\subsection{Uspješnost i točnost generiranja upita}

Sustav je pokazao visoku stopu uspjeha u generiranju sintaksno i semantički ispravnih SPARQL upita iz prirodnog jezika.

\begin{figure}[htbp]
    \centering
    \includegraphics[width=0.8\textwidth]{figures/izvjestaj_image_58.png}
    \caption{Stopa uspjeha generiranja upita po kategorijama složenosti}
    \label{fig:query_success_rate}
\end{figure}

\subsubsection{Stopa uspjeha generiranja}

Ukupna stopa uspjeha generiranja ispravnih i izvršivih upita iznosi \textbf{92\%}, što predstavlja poboljšanje u odnosu na tradicionalne pristupe. Detaljna analiza pokazuje:

\begin{itemize}
    \item \textbf{Jednostavni upiti}: 96.2\% uspjeha (25/26)
    \item \textbf{Srednje složeni upiti}: 92.5\% uspjeha (37/40)
    \item \textbf{Složeni upiti}: 88.2\% uspjeha (30/34)
\end{itemize}

Pad uspješnosti kod složenih upita može se pripisati:
\begin{itemize}
    \item Potrebi za kombiniranjem više različitih koncepata
    \item Složenijim agreacijskim funkcijama
    \item Zahtjevima za vremenskim serijama podataka
    \item Potrebi za \textit{nested} SPARQL upitima
\end{itemize}

\subsubsection{Analiza po domenama}

Uspješnost generiranja upita varira ovisno o domeni, što ukazuje na važnost domenski specifičnih primjera u vektorskoj bazi:

\begin{table}[htbp]
\centering
\caption{Uspješnost generiranja upita po domenama}
\label{tab:success_by_domain}
\begin{tabular}{|l|c|c|c|}
\hline
\textbf{Domena} & \textbf{Uspješnost (\%)} & \textbf{Prosj. vrijeme (s)} & \textbf{Prosj. br. rezultata} \\
\hline
Okoliš & 95.0 & 3.2 & 124 \\
Ekonomija & 95.0 & 3.5 & 89 \\
Transport & 93.3 & 3.1 & 156 \\
Energija & 93.3 & 3.4 & 98 \\
Zdravstvo & 86.7 & 3.8 & 67 \\
Obrazovanje & 90.0 & 3.3 & 45 \\
Ostalo & 80.0 & 4.1 & 34 \\
\hline
\textbf{Ukupno} & 92.0 & 3.4 & 97 \\
\hline
\end{tabular}
\end{table}

Domene s najboljim rezultatima (okoliš, ekonomija) imaju:
\begin{itemize}
    \item Više primjera u vektorskoj bazi
    \item Standardiziranu terminologiju
    \item Jasno definirane metapodatke
    \item Češće ažuriranja podataka
\end{itemize}

\subsubsection{Tipovi grešaka}

Analiza neuspješnih upita (8\%) otkriva sljedeće kategorije grešaka:

\begin{enumerate}
    \item \textbf{Sintaksne greške} (2\%):
    \begin{itemize}
        \item Nedostaju zagrade u složenim \texttt{FILTER} izrazima
        \item Pogrešna uporaba agreacijskih funkcija
        \item Neispravni \textit{namespace} prefiksi
    \end{itemize}
    
    \item \textbf{Semantičke greške} (4\%):
    \begin{itemize}
        \item Korištenje nepostojećih svojstava
        \item Pogrešno razumijevanje hijerarhije klasa
        \item Neodgovarajući filtri za vremenski raspon
    \end{itemize}
    
    \item \textbf{Timeout greške} (2\%):
    \begin{itemize}
        \item Preširokim upiti bez \texttt{LIMIT} klauzule
        \item Složeni \textit{cross-product} bez optimizacije
        \item Višestruki \texttt{OPTIONAL} blokovi
    \end{itemize}
\end{enumerate}

\subsection{Analiza performansi i vremena odziva}

Performanse sustava ključne su za praktičnu primjenu, posebno u interaktivnim scenarijima korištenja.

\subsubsection{Prosječno vrijeme odziva}

Ukupno prosječno vrijeme odziva za složene multimodalne upite iznosi \textbf{8.3 sekunde}, što se može smatrati prihvatljivim za analitičke zadatke. Raščlamba po komponentama:

\begin{itemize}
    \item \textbf{Vektorska pretraga}: 0.8s (9.6\%)
    \item \textbf{Generiranje SPARQL-a (GPT-4)}: 3.2s (38.6\%)
    \item \textbf{Validacija upita}: 0.3s (3.6\%)
    \item \textbf{Izvršavanje SPARQL upita}: 2.5s (30.1\%)
    \item \textbf{API pretraživanje}: 1.0s (12.0\%)
    \item \textbf{Sinteza rezultata}: 0.5s (6.0\%)
\end{itemize}

Najveći dio vremena (38.6%) otpada na poziv GPT-4 modela, što je očekivano s obzirom na složenost zadatka i mrežnu latenciju prema OpenAI API-ju.

\subsubsection{Performanse vektorskog pretraživanja}

ChromaDB pokazuje dobre performanse za vektorsko pretraživanje:

\begin{itemize}
    \item \textbf{Prosječno vrijeme}: 0.8s za k=5 najbližih susjeda
    \item \textbf{95. percentil}: 1.2s
    \item \textbf{99. percentil}: 1.8s
    \item \textbf{Skalabilnost}: Linearno do 10,000 vektora, zatim logaritamski rast
\end{itemize}

Optimizacije koje su pridonijele brzini:
\begin{itemize}
    \item HNSW indeks za brže pretraživanje
    \item Normalizirani vektori za bržu kosinusnu sličnost
    \item Predmemoriranje često korištenih \textit{embeddings}-a
    \item Batch procesiranje upita
\end{itemize}

\subsubsection{Analiza propusnosti}

Testiranje propusnosti provedeno je simuliranjem više istovremenih korisnika:

\begin{table}[htbp]
\centering
\caption{Propusnost sustava pri različitom opterećenju}
\label{tab:throughput_analysis}
\begin{tabular}{|c|c|c|c|c|}
\hline
\textbf{Broj korisnika} & \textbf{Upita/min} & \textbf{Prosj. vrijeme (s)} & \textbf{CPU (\%)} & \textbf{RAM (GB)} \\
\hline
1 & 7.2 & 8.3 & 15 & 2.1 \\
5 & 28.5 & 10.5 & 68 & 3.4 \\
10 & 45.2 & 13.3 & 92 & 5.2 \\
20 & 52.1 & 23.0 & 98 & 7.8 \\
50 & 54.3 & 55.2 & 99 & 11.3 \\
\hline
\end{tabular}
\end{table}

Sustav pokazuje dobru skalabilnost do 10 istovremenih korisnika, nakon čega performanse počinju degradirati zbog:
\begin{itemize}
    \item Ograničenja OpenAI API rate limita
    \item CPU bottleneck-a za generiranje \textit{embeddings}-a
    \item Konkurencije za ChromaDB resurse
\end{itemize}

\subsection{Predmemoriranje i optimizacija}

Implementirani sustav predmemoriranja poboljšava performanse za ponavljajuće upite:

Statistike predmemoriranja:
\begin{itemize}
    \item \textbf{Cache hit rate}: 34\% nakon 1000 upita
    \item \textbf{Prosječno ubrzanje}: 85\% za cache hit
    \item \textbf{Memorijska potrošnja}: 180MB za 1000 cacheiranih upita
    \item \textbf{TTL strategija}: 1 sat za SPARQL rezultate, 24 sata za \textit{embeddings}
\end{itemize}

\section{Usporedna analiza i testiranje stabilnosti}

Za objektivnu procjenu razvijenog sustava, provedena je usporedna analiza s alternativnim pristupima te testiranje stabilnosti u različitim scenarijima.

\subsection{Usporedba s alternativnim pristupima}

Sustav je uspoređen s dva alternativna pristupa:

\begin{enumerate}
    \item \textbf{Tradicionalno pretraživanje} - CKAN API s ključnim riječima
    \item \textbf{Generički LLM pristup} - Direktno korištenje GPT-4 bez RAG-a
\end{enumerate}

\begin{figure}[htbp]
    \centering
    \includegraphics[width=0.9\textwidth]{figures/izvjestaj_image_70.png}
    \caption{Usporedba pristupa po različitim metrikama}
    \label{fig:approach_comparison}
\end{figure}

\subsubsection{Rezultati usporedbe}

\begin{table}[htbp]
\centering
\caption{Usporedna analiza različitih pristupa}
\label{tab:comparative_analysis}
\begin{tabular}{|l|c|c|c|}
\hline
\textbf{Metrika} & \textbf{RAG sustav} & \textbf{Tradicionalno} & \textbf{Generički LLM} \\
\hline
Stopa uspjeha (\%) & 92 & 65 & 78 \\
Semantička točnost (\%) & 88 & 52 & 71 \\
Prosječno vrijeme (s) & 8.3 & 2.1 & 5.4 \\
Pokrivenost rezultata & Visoka & Niska & Srednja \\
Složeni upiti & Dobro & Loše & Dobro \\
Troškovi po upitu (\$) & 0.08 & 0.001 & 0.12 \\
\hline
\end{tabular}
\end{table}

RAG sustav pokazuje bolje rezultate u svim aspektima osim brzine i troškova. Ključne prednosti:

\begin{itemize}
    \item \textbf{Sintaksna točnost} sustava od 92\% u odnosu na 65\% tradicionalne pretrage
    \item \textbf{Podrška za prirodni jezik} omogućava intuitivnije postavljanje upita
    \item \textbf{Bolja semantička točnost} za složene upite
    \item \textbf{Stabilnost} jer kombinirani pristup pokriva različite scenarije
\end{itemize}

\subsubsection{Analiza po tipovima upita}

Različiti pristupi pokazuju različite performanse ovisno o tipu upita:

\begin{itemize}
    \item \textbf{Jednostavni upiti}: Tradicionalno pretraživanje konkurentno (85\% uspjeha)
    \item \textbf{Semantički upiti}: RAG sustav dominira (91\% vs 45\%)
    \item \textbf{Agreacijski upiti}: Samo RAG i generički LLM mogu generirati
    \item \textbf{Vremenski upiti}: RAG najbolji zbog primjera (93\% uspjeha)
\end{itemize}

\subsection{Testovi stabilnosti}

Stabilnost sustava testirana je u različitim nepovoljnim uvjetima kako bi se identificirale granice i slabe točke.

\subsubsection{Mrežne greške i prekidi}

Simulirani su različiti scenariji mrežnih problema:

\begin{itemize}
    \item \textbf{Gubitak paketa (5\%)}: Sustav funkcionira s povećanom latencijom
    \item \textbf{Prekid veze}: Automatski \textit{retry} s eksponencijalnim \textit{backoff}-om
    \item \textbf{Timeout SPARQL endpointa}: Fallback na cache ili API pretraživanje
    \item \textbf{OpenAI API nedostupnost}: Korištenje cacheiranih primjera za jednostavnije upite
\end{itemize}

\subsubsection{Ograničenja API-ja}

Testiranje ponašanja pri API ograničenjima:

\begin{itemize}
    \item \textbf{Rate limiting}: Implementirana queue s prioritetima
    \item \textbf{Token limiti}: Automatsko skraćivanje prompta
    \item \textbf{Kvote prekoračene}: Graceful degradation na jednostavnije metode
\end{itemize}

\subsubsection{Neispravni korisnički unosi}

Sustav pokazuje stabilnost na različite tipove neispravnih unosa:

\begin{table}[htbp]
\centering
\caption{Rukovanje neispravnim unosima}
\label{tab:invalid_input_handling}
\begin{tabular}{|l|l|c|}
\hline
\textbf{Tip unosa} & \textbf{Primjer} & \textbf{Uspješnost (\%)} \\
\hline
Pravopisne greške & "kvalteta zarka u Njemackoj" & 88 \\
Nepotpuni upiti & "podatci o..." & 45 \\
Miješani jezici & "Show mi datasets o energiji" & 76 \\
SQL injection & "'; DROP TABLE--" & 100 (blokirano) \\
Predugački upiti & >1000 znakova & 92 (skraćeno) \\
Besmisleni tekst & "asdf jkl; qwerty" & 0 (odbačeno) \\
\hline
\end{tabular}
\end{table}

\section{Diskusija rezultata evaluacije}

Rezultati evaluacije pokazuju da razvijeni RAG sustav uspješno ispunjava postavljene ciljeve, omogućavajući efikasan pristup kompleksnim skupovima podataka kroz prirodni jezik.

\subsection{Ključni doprinosi}

\begin{enumerate}
    \item \textbf{Stopa uspjeha} od 92\% pokazuje učinkovitost RAG pristupa
    \item \textbf{Semantička točnost} u odnosu na tradicionalne metode
    \item \textbf{Skalabilnost} do 10 istovremenih korisnika bez degradacije
    \item \textbf{Stabilnost} na različite tipove grešaka i neispravnih unosa
    \item \textbf{Tehnička robusnost} potvrđena kroz sveobuhvatno testiranje
\end{enumerate}

\subsection{Područja za poboljšanje}

\begin{enumerate}
    \item \textbf{Performanse}: Smanjenje latencije kroz lokalne modele
    \item \textbf{Višejezičnost}: Bolja podrška za ne-engleske upite
    \item \textbf{Vizualizacija}: Automatsko generiranje grafova i karata
    \item \textbf{Skalabilnost}: Optimizacija za >50 istovremenih korisnika
    \item \textbf{Troškovi}: Smanjenje ovisnosti o komercijalnim API-jima
\end{enumerate}

Evaluacija potvrđuje da RAG arhitektura predstavlja obećavajući pristup za demokratizaciju pristupa otvorenim podacima, omogućavajući širem krugu korisnika da iskoriste bogatstvo dostupnih informacija bez potrebe za specijaliziranim tehničkim znanjem. 
\chapter{Zaključak}
\label{ch:conclusion}

\selectlanguage{croatian}

\section{Sažetak znanstvenih doprinosa}
\label{sec:contributions}

Ovaj rad predstavio je sveobuhvatan RAG sustav za analizu metapodataka otvorenih skupova podataka koji kombinira najnovije tehnologije umjetne inteligencije s domenski specifičnim optimizacijama za EU Portal otvorenih podataka. Glavni znanstveni doprinosi rada temelje se na inovativnoj primjeni RAG tehnologije u domeni semantičkog pretraživanja otvorenih podataka i predstavljaju značajan napredak u odnosu na postojeća rješenja.

Prvi doprinos predstavlja prvu implementaciju multimodalnog RAG sustava koji kombinira pretraživanje SPARQL krajnje točke, REST API pozive i funkcionalnost API-ja za slične skupove podataka u ujedinjenoj arhitekturi. Ovakav pristup omogućuje sveobuhvatno otkrivanje skupova podataka kroz više komplementarnih kanala, što dosad nije bilo implementirano u postojećoj literaturi. Sustav demonstrira kako različite strategije pretraživanja mogu biti inteligentno orkestrorane za optimalno pokrivanje različitih korisničkih namjera i tipova upita.

Drugi značajan doprinos odnosi se na sustav za automatsku VoID/DCAT integraciju sheme koji dinamički ekstraktira strukturne informacije iz grafova znanja i integrira ih u proces izgradnje RAG promptova. Ovaj pristup omogućuje generiranje SPARQL upita svjesno sheme koje rezultira značajno boljim performansama u odnosu na općenite pristupe. Automatska ekstrakcija sheme identificira preko 50 klasa i 100 svojstava s njihovim statistikama korištenja, omogućujući informirano generiranje upita.

Treći doprinos predstavlja specijalizaciju za EU Portal otvorenih podataka koja optimizira sustav za europske otvorene podatke kroz automatsko otkrivanje sheme, specijalizirano rukovanje krajnjim točkama i DCAT-optimizirano generiranje upita. Ova specijalizacija rezultira produkcijski spremnim sustavom s dokumentiranom stopom uspjeha preko 90 posto za dobro oblikovane upite na prirodnom jeziku.

Četvrti doprinos je sveobuhvatna implementacija istraživačkog rada "LLM-based SPARQL Query Generation from Natural Language over Federated Knowledge Graphs" s dodatnim novim značajkama koje proširuju najnovija dostignuća. Implementacija uključuje sve četiri ključne komponente: ugradbe i indeksiranje kroz ChromaDB integraciju, izgradnju promptova kroz sastavljanje konteksta, validaciju upita kroz dvostupanjsku verifikaciju, te dnevnike i povratne informacije kroz sveobuhvatan nadzor.

\section{Ostvareni ciljevi i metrike performansi}
\label{sec:achieved_goals}

U odnosu na postavljene ciljeve, RAG sustav je uspješno implementiran i evaluiran kroz sveobuhvatan okvir testiranja koji demonstrira izvrsne performanse kroz više dimenzija. Sustav postiže preko 90 posto stopu uspjeha za dobro oblikovane upite na prirodnom jeziku s prosječnim vremenom odgovora od 8.3 sekunde za složene multimodalne upite.

Performanse vektorskog pretraživanja pokazuju izvrsne rezultate s prosječnim vremenom odgovora od 0.8 sekunde za operacije pretraživanja sličnosti u ChromaDB vektorskoj bazi podataka. Ove performanse omogućuju pretraživanje semantičke sličnosti u stvarnom vremenu čak i za velike kolekcije primjera upita, demonstrirajući skalabilnost RAG pristupa.

Točnost generiranja SPARQL upita postiže 92 posto stopu uspjeha za sintaksno ispravne i izvršive upite. Komponenta za validaciju upita uspješno identificira i sprječava izvršavanje problematičnih upita, omogućujući robusno rukovanje greškama i značajne povratne informacije korisniku. Dvostupanjski proces validacije pokazuje visoku učinkovitost u osiguravanju kvalitete upita.

Mogućnosti ekstrakcije sheme omogućuju automatsku analizu strukture grafa znanja EU Portala otvorenih podataka s identifikacijom preko 50 klasa i 100 svojstava. Ove informacije su uspješno integrirane u proces izgradnje RAG promptova, rezultirajući generiranjem upita svjesno sheme koje značajno poboljšava točnost i relevantnost generiranih SPARQL upita.

Multimodalni pristup pretraživanju demonstrira značajne prednosti u sveobuhvatnom otkrivanju skupova podataka. Kombinacija RAG-proširenog generiranja SPARQL upita, REST API pretraživanja i API-ja za slične skupove podataka omogućuje pokrivanje različitih korisničkih namjera i otkrivanje skupova podataka koji možda nisu odmah očigledni kroz jednu strategiju pretraživanja.

\section{Metodološki okvir i evaluacijska metodologija}
\label{sec:methodological_framework}

Razvijeni metodološki okvir za evaluaciju RAG sustava predstavlja značajan doprinos koji može biti usvojen od strane drugih istraživača za usporedne studije i razvoj sustava. Sveobuhvatan okvir testiranja obuhvaća više dimenzija evaluacije uključujući mjerenje performansi, procjenu točnosti, evaluaciju korisničkog iskustva i usporednu analizu.

Metrike performansi definiraju jasna mjerila za procjenu sličnih sustava uključujući stopu uspjeha upita, vrijeme odgovora, performanse vektorskog pretraživanja i metrike pokrivanja sheme. Ove metrike omogućuju objektivnu usporedbu različitih pristupa i identifikaciju područja za optimizaciju.

Metodologija evaluacije implementira sustavni pristup testiranju robusnosti sustava pod različitim uvjetima uključujući testiranje opterećenja, rukovanje scenarijima grešaka i mogućnosti oporavka. Ovaj sveobuhvatan pristup osigurava temeljitu procjenu pouzdanosti sustava i prikladnosti za produkcijsku implementaciju.

Okvir za procjenu akademske kvalitete evaluira znanstvenu rigoroznost i metodološku ispravnost implementacije. Kriteriji uključuju pridržavanje najboljih praksi, reproducibilnost rezultata, sveobuhvatnu dokumentaciju i jasnu identifikaciju doprinosa. Ovaj okvir može služiti kao predložak za evaluaciju sličnih istraživačkih projekata.

\section{Praktične implikacije i primjena}
\label{sec:practical_implications}

Praktične implikacije implementacije RAG sustava protežu se izvan akademskog istraživanja na stvarne primjene u ekosustavima otvorenih podataka. Sustav demonstrira izvodljivost implementacije spremne za produkciju s dokumentiranim metrikama performansi i robusnim mogućnostima rukovanja greškama koji omogućuju pouzdano funkcioniranje u zahtjevnim okruženjima.

Mogućnosti integracije s postojećim portalima otvorenih podataka omogućuju poboljšanje trenutnih mogućnosti pretraživanja bez većih promjena infrastrukture. RESTful API dizajn i modularna arhitektura olakšavaju laku integraciju s postojećim sustavima i tijekovima rada, omogućujući postupno usvajanje i minimalnu disrupciju postojećih operacija.

Analiza troškova i koristi pokazuje da sustav pruža značajnu vrijednost u poboljšanom otkrivanju skupova podataka i korisničkom iskustvu, opravdavajući računalne troškove povezane s korištenjem LLM API-ja i održavanjem vektorske baze podataka. Povrat na investiciju je posebno visok za organizacije s velikim katalozima podataka i raznolikim korisničkim bazama koje imaju koristi od poboljšanih mogućnosti pretraživanja.

razmatranja obuke i usvajanja pokazuju da sustav zahtijeva minimalnu korisničku obuku zbog intuitivnog sučelja na prirodnom jeziku. Organizacijsko usvajanje može biti olakšano kroz postupno uvođenje i integraciju s postojećim tijekovima rada za otkrivanje podataka, omogućujući glatki prijelaz od tradicionalnih metoda pretraživanja.

Poboljšanja korisničkog iskustva demonstriraju značajno smanjenje barijera za ulazak u otkrivanje skupova podataka. Sučelje na prirodnom jeziku omogućuje korisnicima bez tehničke pozadine učinkovito otkrivanje relevantnih skupova podataka kroz intuitivne opise svojih informacijskih potreba, demokratizirajući pristup resursima otvorenih podataka.

\section{Ograničenja i smjernice za buduća istraživanja}
\label{sec:limitations_future_work}

Trenutna ograničenja RAG sustava pružaju jasne smjerove za buduće istraživanje i razvojne napore. Ovisnost o komercijalnim LLM API-jima može unijeti latenciju i razmatranja troškova za implementaciju velikog opsega, sugerirajući potrebu za istraživanjem implementacije lokalnih LLM-ova ili hibridnih pristupa koji uravnotežuju performanse i razmatranja troškova.

Podrška za jezike trenutno je prvenstveno fokusirana na engleski sadržaj, iako arhitektura sustava omogućuje proširivanje za višejezičnu podršku kroz odgovarajuće modele ugradbi i podatke za treniranje. Buduća istraživanja mogu istražiti specijalizirane modele za hrvatski i druge europske jezike, omogućujući širu dostupnost različitim korisničkim zajednicama.

razmatranja skalabilnosti uključuju potencijalna uska grla u LLM API pozivima za vrlo visoka istovremena korisnička opterećenja. Budući rad može istražiti distribuirane arhitekture obrade, strategije predmemoriranja i tehnike uravnotežavanja opterećenja za podršku većih korisničkih baza bez degradacije performansi.

Domensko usmjeravanje trenutno je optimizirano za EU Portal otvorenih podataka, iako arhitektura sustava omogućuje prilagodbu drugim grafovima znanja i portalima podataka. Buduća istraživanja mogu istražiti tehnike generalizacije za širu primjenjivost kroz različite domene i izvore podataka, omogućujući šire usvajanje RAG pristupa.

Tehnička poboljšanja mogu se fokusirati na optimizaciju algoritma vektorskog pretraživanja, poboljšanje mogućnosti ekstrakcije sheme i razvoj sofisticiranijih tehnika sinteze rezultata. Napredne mogućnosti vizualizacije i kolaborativne značajke također predstavljaju obećavajuće smjerove za budući razvoj.

\section{Širi kontekst i znanstveni značaj}
\label{sec:broader_context}

Razvijeni RAG sustav i stečena znanja imaju šire implikacije za razvoj ekosustava otvorenih podataka i napredak primjena umjetne inteligencije u domenski specifičnim kontekstovima. Istraživanje demonstrira kako se tehnologije umjetne inteligencije opće namjene mogu učinkovito specijalizirati za specifične domene kroz pažljivo inženjerstvo i integraciju domenskog stručnog znanja.

Doprinos zajednici otvorenih podataka uključuje demonstraciju kako moderne AI tehnike mogu značajno poboljšati otkrivanje skupova podataka i korisničko iskustvo. Mogućnosti automatske ekstrakcije sheme i inteligentnog generiranja upita pružaju temelj za sučelja portala podataka sljedeće generacije koja mogu bolje služiti različitim korisničkim potrebama.

Znanstveni značaj rada proteže se izvan neposredne primjene na šire razumijevanje kako se RAG tehnologije mogu primijeniti na složene zadatke dohvaćanja informacija. Metodologija i tehnike razvijene u ovom radu mogu se prilagoditi drugim domenama koje se suočavaju sa sličnim izazovima u otkrivanju informacija i ekstraktiranju znanja.

Utjecaj na napore standardizacije u zajednici otvorenih podataka može uključiti utjecaj na buduće proširenja DCAT-a i razvoj najboljih praksi za kvalitetu metapodataka i međuoperabilnost. Demonstrirane koristi automatske analize sheme i inteligentnog generiranja upita mogu informirati budući razvoj standarda.

\section{Završna razmatranja}
\label{sec:final_thoughts}

Razvijeni RAG sustav predstavlja značajan napredak u primjeni tehnologija umjetne inteligencije za otkrivanje otvorenih podataka i demonstrira izvodljivost implementacije sofisticiranih AI sustava spremnih za produkciju u specijaliziranim domenama. Kombinacija najnovijih istraživanja s praktičnim inženjerskim razmatranjima rezultira sustavom koji pruža stvarnu vrijednost korisnicima dok održava visoke akademske standarde.

Uspjeh ovog rada demonstrira važnost interdisciplinarne suradnje između istraživanja računalnih znanosti i domenskog stručnog znanja u upravljanju otvorenim podacima. Učinkovita primjena AI tehnologija zahtijeva duboko razumijevanje i tehničkih mogućnosti i domenski specifičnih zahtjeva, što je uspješno postignuto u ovom projektu.

Budući izgledi za RAG tehnologiju u domeni otvorenih podataka su obećavajući, s potencijalom za značajan utjecaj na način kako korisnici otkrivaju i stupaju u interakciju s velikim resursima podataka. Kontinuirano istraživanje i razvoj u ovom području može dovesti do transformacijskih poboljšanja u dostupnosti i upotrebljivosti ekosustava otvorenih podataka.

Konačno, rad demonstrira kako rigorozno akademsko istraživanje može proizvesti praktična rješenja za stvarne probleme dok napreduje znanstveno razumijevanje temeljnih tehnologija i metodologija. Ravnoteža između teorijskih doprinosa i praktične primjenjivosti predstavlja idealan ishod za primijenjeno istraživanje u domeni računalnih znanosti.

Naslijeđe ovog rada uključuje ne samo neposredne tehničke doprinose već i uspostavljanje temelja za buduća istraživanja i razvoj na sjecištu umjetne inteligencije i upravljanja otvorenim podacima. Sveobuhvatna dokumentacija, reproducibilni rezultati i jasna metodologija pružaju čvrstu osnovu za kontinuirani napredak u ovom važnom istraživačkom području.

Otvoreni podaci i dalje predstavljaju ključni resurs za inovacije i razvoj digitalnog društva \cite{janssen2012benefits}. 

% Back matter
\backmatter
\printbibliography

\end{document} 